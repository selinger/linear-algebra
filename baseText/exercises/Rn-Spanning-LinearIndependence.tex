\Opensolutionfile{solutions}[ex]
\section*{Exercises}

\begin{enumialphparenastyle}

\begin{ex}
  Which of the following vectors are redundant? If there are redundant
  vectors, write each of them as a linear combination of previous
  vectors.
  \begin{equation*}
    \begin{mymatrix}{r} 1 \\ 0 \\ 1 \end{mymatrix},\quad
    \begin{mymatrix}{r} 2 \\ 0 \\ 2 \end{mymatrix},\quad
    \begin{mymatrix}{r} 1 \\ 2 \\ 1 \end{mymatrix},\quad
    \begin{mymatrix}{r} 1 \\ 6 \\ 1 \end{mymatrix}.
  \end{equation*}
  % \begin{sol}
  % \end{sol}
\end{ex}

\begin{ex}
  Which of the following vectors are redundant? If there are redundant
  vectors, write each of them as a linear combination of previous
  vectors.
  \begin{equation*}
    \begin{mymatrix}{r} -1 \\ -2 \\ 2 \\ 3 \end{mymatrix},\quad
    \begin{mymatrix}{r} -3 \\ -4 \\ 3 \\ 3 \end{mymatrix},\quad
    \begin{mymatrix}{r} 0 \\ -1 \\ 4 \\ 3 \end{mymatrix},\quad
    \begin{mymatrix}{r} 0 \\  2 \\ -3 \\ -6 \end{mymatrix}.
  \end{equation*}
  % \begin{sol}
  % \end{sol}
\end{ex}

\begin{ex}
  Use the method of
  Theorem~\ref{thm:characterization-linear-independence} to determine
  whether the following vectors are linearly independent. If they are
  linearly dependent, find a non-trivial linear combination of the
  vectors that is equal to $\vect{0}$.
  \begin{equation*}
    \begin{mymatrix}{r} 2 \\ 3 \\ 2 \end{mymatrix},\quad
    \begin{mymatrix}{r} -1 \\ 0 \\ 2 \end{mymatrix},\quad
    \begin{mymatrix}{r} -3 \\ -4 \\ -2 \end{mymatrix},\quad
    \begin{mymatrix}{r} 5 \\ 6 \\ 2 \end{mymatrix}.
  \end{equation*}
  % \begin{sol}
  % \end{sol}
\end{ex}

\begin{ex}
  Use the method of
  Theorem~\ref{thm:characterization-linear-independence} to determine
  whether the following vectors are linearly independent. If they are
  linearly dependent, find a non-trivial linear combination of the
  vectors that is equal to $\vect{0}$.
  \begin{equation*}
    \begin{mymatrix}{r} 1 \\ -1 \\ 0 \\ 1 \end{mymatrix},\quad
    \begin{mymatrix}{r} 1 \\ 6 \\ 7 \\ 1 \end{mymatrix},\quad
    \begin{mymatrix}{r} 3 \\ 5 \\ 8 \\ 3 \end{mymatrix},\quad
    \begin{mymatrix}{r} 1 \\ 0 \\ 1 \\ 1 \end{mymatrix}.
  \end{equation*}
  % \begin{sol}
  % \end{sol}
\end{ex}

\begin{ex}
  Are the following vectors linearly independent? If not, write one of
  them as a linear combination of the others.
  \begin{equation*}
    \begin{mymatrix}{r} 1 \\ 3 \\  1 \end{mymatrix},\quad
    \begin{mymatrix}{r} 1 \\ 4 \\  2 \end{mymatrix},\quad
    \begin{mymatrix}{r} 1 \\ 1 \\ -1 \end{mymatrix}.
  \end{equation*}
  % \begin{sol}
  % \end{sol}
\end{ex}

\begin{ex}
  Find a linearly independent set of vectors that has the same span as
  the given vectors.
  \begin{equation*}
    \begin{mymatrix}{r} 2 \\  0 \\  3 \end{mymatrix},\quad
    \begin{mymatrix}{r} 1 \\  3 \\  5 \end{mymatrix},\quad
    \begin{mymatrix}{r} 3 \\  3 \\  8 \end{mymatrix},\quad
    \begin{mymatrix}{r} 3 \\ -3 \\  1 \end{mymatrix}.
  \end{equation*}
  % \begin{sol}
  % \end{sol}
\end{ex}

\begin{ex}
  Find a linearly independent set of vectors that has the same span as
  the given vectors.
  \begin{equation*}
    \begin{mymatrix}{r} 1 \\ 3 \\  3 \\ 1 \end{mymatrix},\quad
    \begin{mymatrix}{r} 2 \\ 6 \\  6 \\ 2 \end{mymatrix},\quad
    \begin{mymatrix}{r} 1 \\ 0 \\ -3 \\ 1 \end{mymatrix},\quad
    \begin{mymatrix}{r} 1 \\ 2 \\  1 \\ 1 \end{mymatrix}.
  \end{equation*}
  % \begin{sol}
  % \end{sol}
\end{ex}

\begin{ex}
  Here are some vectors in $\R^{4}$.
  \begin{equation*}
    \begin{mymatrix}{r} 1 \\ 1 \\ -1 \\ 1 \end{mymatrix},\quad
    \begin{mymatrix}{r} 1 \\ 2 \\ -1 \\ 1 \end{mymatrix},\quad
    \begin{mymatrix}{r} 1 \\ -2 \\ -1 \\ 1 \end{mymatrix},\quad
    \begin{mymatrix}{r} 1 \\ 2 \\ 0 \\ 1 \end{mymatrix},\quad
    \begin{mymatrix}{r} 1 \\ -1 \\ -1 \\ 1 \end{mymatrix}.
  \end{equation*}
  Explain why these vectors can't possibly be linearly
  independent. Then obtain a linearly independent subset of these
  vectors that has the same span as these vectors.
  % \begin{sol}
  % \end{sol}
\end{ex}

\begin{ex}
  Here are some vectors in $\R^{4}$.
  \begin{equation*}
    \begin{mymatrix}{r} 1 \\ -1 \\ -1 \\ 1 \end{mymatrix},\quad
    \begin{mymatrix}{r} -3 \\ 3 \\ 3 \\ -3 \end{mymatrix},\quad
    \begin{mymatrix}{r} 1 \\ 0 \\ -1 \\ 1 \end{mymatrix},\quad
    \begin{mymatrix}{r} 2 \\ -9 \\ -2 \\ 2 \end{mymatrix},\quad
    \begin{mymatrix}{r} 1 \\ 0 \\ 0 \\ 1 \end{mymatrix}.
  \end{equation*}
  Explain why these vectors can't possibly be linearly
  independent. Then find a non-trivial linear combination of these
  vectors that equals $\vect{0}$.
  % \begin{sol}
  % \end{sol}
\end{ex}

\begin{ex}
  Here are some vectors.
  \begin{equation*}
    \begin{mymatrix}{r} 1 \\ 1 \\ -2 \end{mymatrix},\quad
    \begin{mymatrix}{r} 1 \\ 2 \\ -2 \end{mymatrix},\quad
    \begin{mymatrix}{r} 2 \\ 7 \\ -4 \end{mymatrix},\quad
    \begin{mymatrix}{r} 5 \\ 7 \\ -10 \end{mymatrix},\quad
    \begin{mymatrix}{r} 12 \\ 17 \\ -24 \end{mymatrix}.
  \end{equation*}
  Describe the span of these vectors as the span of as few vectors as
  possible.
  % \begin{sol}
  % \end{sol}
\end{ex}

\begin{ex}
  Here are some vectors.
  \begin{equation*}
    \begin{mymatrix}{r} 1 \\ 2 \\ -2 \end{mymatrix},\quad
    \begin{mymatrix}{r} 1 \\ 3 \\ -2 \end{mymatrix},\quad
    \begin{mymatrix}{r} 1 \\ -2 \\ -2 \end{mymatrix},\quad
    \begin{mymatrix}{r} -1 \\ 0 \\ 2 \end{mymatrix},\quad
    \begin{mymatrix}{r} 1 \\ 3 \\ -1 \end{mymatrix}.
  \end{equation*}
  Describe the span of these vectors as the span of as few vectors as
  possible.
  % \begin{sol}
  % \end{sol}
\end{ex}

\begin{ex}
  Let $\vect{u},\vect{v},\vect{w}$ be linearly independent vectors in
  $\R^n$. Are the vectors $\vect{u}+\vect{v}$, $2\vect{u}+\vect{w}$,
  and $\vect{w}-2\vect{v}$ linearly independent?
\end{ex}

\begin{ex}
  Let $\vect{u},\vect{v},\vect{w}$ be linearly independent vectors in
  $\R^n$. Are the vectors $\vect{u}+\vect{v}$, $\vect{u}+\vect{w}$,
  and $\vect{w}+\vect{v}$ linearly independent?
\end{ex}

\end{enumialphparenastyle}
