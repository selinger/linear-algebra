\section*{Exercises}

\begin{ex}
  For the following pairs of matrices, determine if the sum $A+B$
  and the difference $A-B$ are defined. If so, calculate them.
  \begin{enumerate}
  \item
    $A = \begin{mymatrix}{rr}
      1 & 0 \\
      0 & 1
    \end{mymatrix}$,\quad
    $B = \begin{mymatrix}{rr}
      0 & 1 \\
      1 & 0
    \end{mymatrix}$.

  \item
    $A = \begin{mymatrix}{rrr}
      2 & 1 & 2 \\
      1 & 1 & 0
    \end{mymatrix}$,\quad
    $B = \begin{mymatrix}{rrr}
      -1 & 0 & 3 \\
      0 & 1 & 4
    \end{mymatrix}$.

  \item
    $A = \begin{mymatrix}{rr}
      1 & 0 \\
      -2 & 3 \\
      4 & 2
    \end{mymatrix}$,\quad
    $B = \begin{mymatrix}{rrr}
      2 & 7 & -1 \\
      0 & 3 & 4
    \end{mymatrix}$.
  \end{enumerate}
\end{ex}

\begin{ex}
  For each matrix $A$, find the matrix $-A$.
  \begin{enumerate}
  \item
    $A = \begin{mymatrix}{rr}
      1 & 2 \\
      2 & 1
    \end{mymatrix}$

  \item
    $A = \begin{mymatrix}{rr}
      -2 & 3 \\
      0 & 2
    \end{mymatrix}$

  \item
    $A = \begin{mymatrix}{rrr}
      0 & 1 & 2 \\
      1 & -1 & 3 \\
      4 & 2 & 0
    \end{mymatrix}$
  \end{enumerate}
  % \begin{sol}
  % \end{sol}
\end{ex}

\begin{ex}
  Let $A = \begin{mymatrix}{rrr}
    1 & 2 & -1 \\
    -1 & 4 & 0 \\
  \end{mymatrix}$ and
  $B = \begin{mymatrix}{rrr}
    0 & 3 & 0 \\
    1 & -1 & 1 \\
  \end{mymatrix}$.\par\noindent
  Find a matrix $X$ such that $(A+X)-(B+0) = B+A$. Hint: first use
  the properties of matrix addition to simplify the equation and
  solve for $X$.
  \begin{sol}
    The equation simplifies to $X=B+B$, so $X = \begin{mymatrix}{rrr}
    0 & 6 & 0 \\
    2 & -2 & 2 \\
    \end{mymatrix}$.
  \end{sol}
\end{ex}

\begin{ex}\label{add-inv-rst-unique}
  Using only the properties given in
  Proposition~\ref{prop:properties-of-addition}, show that if $A+B=0$, then
  $B=-A$.
  \begin{sol}
    Suppose $A+B=0$. Then we have
    $-A = (-A)+0 = (-A)+(A+B) = ((-A)+A)+B = (A+(-A))+B = 0+B = B+0 =
    B$.  Here, we have used the additive unit law, the assumption
    $A+B=0$, associativity, commutativity, the additive inverse law,
    commutativity, and the additive unit law, in that order.
  \end{sol}
\end{ex}

\begin{ex} Using only the properties given in
  Proposition~\ref{prop:properties-of-addition}, show $A+B=A$ implies $B=0$.
  \begin{sol}
    Suppose $A+B=A$. Then $B=B+0=0+B=(A+(-A))+B = ((-A)+A)+B =
    (-A)+(A+B) = (-A)+A = A+(-A) = 0$. Here, we have used the additive
    unit law, commutativity, the additive inverse law, commutativity,
    associativity, the assumption $A+B=A$, commutativity, and the
    additive inverse law, in that order.
  \end{sol}
\end{ex}

