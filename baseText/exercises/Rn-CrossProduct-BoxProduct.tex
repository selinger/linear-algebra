\begin{enumialphparenastyle}

\begin{ex} Find the volume of the parallelepiped determined by the vectors
$\begin{mymatrix}{r}
1 \\
-7 \\
-5
\end{mymatrix} $, \\
 $\begin{mymatrix}{r}
1 \\
-2 \\
-6
\end{mymatrix}$, and $\begin{mymatrix}{r}
3 \\
2 \\
3
\end{mymatrix}$.
\begin{sol}
 $\begin{absmatrix}{rrr}
1 & -7 & -5 \\
1 & -2 & -6 \\
3 & 2 & 3
\end{absmatrix}= 113$
\end{sol}
\end{ex}

\begin{ex} Suppose $\vect{u},\vect{v}$, and $\vect{w}$ are three vectors whose
components are all integers. Can you conclude the volume of the
parallelepiped determined from these three vectors will always be an integer?
\begin{sol}
Yes. It will involve the sum of product of integers and so it will
be an integer.
\end{sol}
\end{ex}

\begin{ex} \label{exer-box-product-zero}
  What does it mean geometrically if the box product of three vectors
  equals zero?
  \begin{sol}
    It means that if you place them so that they all have their tails
    at the same point, the three will lie in the same plane.
  \end{sol}
\end{ex}

\begin{ex} Show that 
  \begin{equation*}
    \tup{\vect{u}\times \vect{v}} \dotprod \vect{w}=
    \tup{\vect{v}\times \vect{w}} \dotprod \vect{u}=
    \tup{\vect{w}\times \vect{u}} \dotprod \vect{v}.
  \end{equation*}
  % \begin{sol}
  % \end{sol}
\end{ex}

\begin{ex} Simplify $\tup{\vect{u}\times \vect{v}} \dotprod \tup{\tup{
\vect{v}\times \vect{w}} \times \tup{\vect{w}\times \vect{z}}} .$
\begin{sol}
Here $\mat{\vect{v},\vect{w},\vect{z}}$ denotes the box product. Consider the cross product term. From the above,
\begin{eqnarray*}
\tup{\vect{v}\times \vect{w}} \times \tup{\vect{w}\times \vect{z}} &=& 
\mat{\vect{v},\vect{w},\vect{z}} \vect{w}-\mat{\vect{w},\vect{w},\vect{z}} \vect{v} \\
&=&\mat{\vect{v},\vect{w},\vect{z}} \vect{w}
\end{eqnarray*}
Thus it reduces to
\[
\tup{\vect{u}\times \vect{v}} \dotprod \mat{\vect{v},\vect{w},\vect{z}} \vect{w}=\mat{\vect{v},\vect{w},\vect{z}} \mat{\vect{u},\vect{v},\vect{w}}
\]
\end{sol}
\end{ex}

\begin{ex} Simplify $\norm{\vect{u}\times \vect{v}} ^{2}+\tup{
\vect{u}\dotprod \vect{v}} ^{2}-\norm{\vect{u}} ^{2}\norm{
\vect{v}} ^{2}.$
\begin{sol}
\begin{eqnarray*}
\norm{\vect{u}\times \vect{v}} ^{2} &=&\varepsilon
_{ijk}u_{j}v_{k}\varepsilon _{irs}u_{r}v_{s}=\tup{\delta _{jr}\delta
_{ks}-\delta _{kr}\delta _{js}} u_{r}v_{s}u_{j}v_{k} \\
&=&u_{j}v_{k}u_{j}v_{k}-u_{k}v_{j}u_{j}v_{k}=\norm{\vect{u}
} ^{2}\norm{\vect{v}} ^{2}-\tup{\vect{u}\dotprod \vect{v}} ^{2}
\end{eqnarray*}
It follows that the expression reduces to $0$. You can also do the following.
\begin{eqnarray*}
\norm{\vect{u}\times \vect{v}} ^{2} &=&\norm{\vect{u}
} ^{2}\norm{\vect{v}} ^{2}\sin ^{2}\theta \\
&=&\norm{\vect{u}} ^{2}\norm{\vect{v}}
^{2}\tup{1-\cos ^{2}\theta } \\
&=&\norm{\vect{u}} ^{2}\norm{\vect{v}}
^{2}-\norm{\vect{u}} ^{2}\norm{\vect{v}}
^{2}\cos ^{2}\theta \\
&=&\norm{\vect{u}} ^{2}\norm{\vect{v}}
^{2}-\tup{\vect{u}\dotprod \vect{v}} ^{2}
\end{eqnarray*}
which implies the expression equals $0$.
\end{sol}
\end{ex}

\begin{ex} This problem uses calculus. For
  $\vect{u},\vect{v},\vect{w}$ functions of $t$, prove that the
  derivative satisfies the following product rules:
  \begin{eqnarray*}
    \tup{\vect{u}\times \vect{v}} ^{\prime } &=&\vect{u}^{\prime }\times
                                                 \vect{v}+\vect{u}\times \vect{v}^{\prime } \\
    \tup{\vect{u}\dotprod \vect{v}} ^{\prime } &=&\vect{u}^{\prime }\dotprod
                                                   \vect{v}+\vect{u}\dotprod \vect{v}^{\prime }
  \end{eqnarray*}
  \begin{sol}
    We will show it using the summation convention and permutation symbol.
    \begin{eqnarray*}
      \tup{\tup{\vect{u}\times \vect{v}} ^{\prime }} _{i} &=
      &\tup{\tup{\vect{u}\times \vect{v}} _{i}} ^{\prime }=\tup{
        \varepsilon _{ijk}u_{j}v_{k}} ^{\prime } \\
                                                          &=&\varepsilon _{ijk}u_{j}^{\prime }v_{k}+\varepsilon
                                                              _{ijk}u_{k}v_{k}^{\prime }=\tup{\vect{u}^{\prime } \times
                                                              \vect{v}+\vect{u}\times \vect{v}^{\prime }} _{i}
    \end{eqnarray*}
    and so $\tup{\vect{u}\times \vect{v}} ^{\prime }=\vect{u}^{\prime }
    \times \vect{v}+\vect{u}\times \vect{v}^{\prime }.$ 
  \end{sol}
\end{ex}

\end{enumialphparenastyle}
