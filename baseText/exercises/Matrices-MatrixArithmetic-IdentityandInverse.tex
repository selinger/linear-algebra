\section*{Exercises}

\begin{enumialphparenastyle}

\begin{ex}
  Suppose $AB=AC$ and $A$ is an invertible $n\times n$-matrix. Does it
  follow that $B=C$? Explain why or why not.
  \begin{sol}
    Yes $B=C$. Multiply $AB = AC$ on the left by $A^{-1}$. 
  \end{sol}
\end{ex}

\begin{ex}
  Suppose $AB=AC$ and $A$ is a non invertible $n\times n$-matrix. Does
  it follow that $B=C$? Explain why or why not.
  \begin{sol}
    No. For example, let
    $A=\begin{mymatrix}{cc}1&1\\1&1\end{mymatrix}$,
    $B=\begin{mymatrix}{cc}1&2\\3&4\end{mymatrix}$, and
    $C=\begin{mymatrix}{cc}2&3\\2&3\end{mymatrix}$. Then $AB=AC$ but
    $B\neq C$.
  \end{sol}
\end{ex}

\begin{ex}
  Give an example of a matrix $A$ such that $A^{2}=I$ and yet
  $A\neq I$ and $A\neq -I$.
  \begin{sol}
    $A = \begin{mymatrix}{rrr}
      1 & 0 & 0 \\
      0 & -1 & 0 \\
      0 & 0 & 1
    \end{mymatrix} $
  \end{sol}
\end{ex}

\end{enumialphparenastyle}
