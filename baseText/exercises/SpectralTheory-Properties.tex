\begin{ex}
  Suppose $A$ is an $n\times n$-matrix and let $\vect{v}$ be an
  eigenvector such that $A\vect{v}=\eigenvar \vect{v}$. Also suppose the
  characteristic polynomial of $A$ is
  \begin{equation*}
    \det (\eigenvar  I-A) =\eigenvar  ^{n}+a_{n-1} \eigenvar  ^{n-1}+\ldots+a_{1}\eigenvar  +a_{0}
  \end{equation*}
  Explain why
  \begin{equation*}
    (A^{n}+a_{n-1}A^{n-1}+\ldots +a_{1}A+a_{0}I) \vect{v}=\vect{0}
  \end{equation*}
  If $A$ is diagonalizable, give a proof of the Cayley Hamilton
  theorem based on this. This theorem says $A$ satisfies its
  characteristic equation,
  \begin{equation*}
    A^{n}+a_{n-1}A^{n-1}+\ldots +a_{1}A+a_{0}I=0
  \end{equation*}
  % \begin{sol}
  % \end{sol}
\end{ex}

\begin{ex}
  Suppose the characteristic polynomial of an $n\times n$-matrix $A$ is
  $1-\eigenvar ^{n}$. Find $A^{mn}$ where $m$ is an integer.
  \begin{sol}
    The eigenvalues are distinct because
    they are the $n\th$ roots of $1$. Hence if $X$ is a given vector with
    \begin{equation*}
      X=\sum_{j=1}^{n}a_{j}V_{j}
    \end{equation*}
    then
    \begin{equation*}
      A^{nm}X=A^{nm}\sum_{j=1}^{n}a_{j}V_{j}=
      \sum_{j=1}^{n}a_{j}A^{nm}V_{j}=\sum_{j=1}^{n}a_{j}V_{j}=X
    \end{equation*}
    so $A^{nm}=I$.
  \end{sol}
\end{ex}

