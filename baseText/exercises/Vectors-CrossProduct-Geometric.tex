\begin{ex}
  Show that if $\vect{a}\times \vect{u}=\vect{0}$ for every unit
  vector $\vect{u}$, then $\vect{a}=\vect{0}$.
  \begin{sol}
    If $\vect{a}\neq \vect{0}$, then the condition says that
    $\norm{\vect{a}\times \vect{u}} = \norm{\vect{a}} \sin \theta =0$
    for all angles $\theta$. Hence $\vect{a} = \vect{0}$.
  \end{sol}
\end{ex}

\begin{ex}
  Find the area of the parallelogram determined by the vectors
  $\begin{mymatrix}{r}
    1 \\
    2 \\
    3
  \end{mymatrix}$, $\begin{mymatrix}{r}
    3 \\
    -2 \\
    1
  \end{mymatrix}$.
  \begin{sol}
    $\begin{mymatrix}{r}
      1 \\
      2 \\
      3
    \end{mymatrix} \times
    \begin{mymatrix}{r}
      3 \\
      -2 \\
      1
    \end{mymatrix} =\begin{mymatrix}{r}
      8 \\
      8 \\
      -8
    \end{mymatrix}$. The area of the parallelogram is $8\sqrt{3}$.
  \end{sol}
\end{ex}

\begin{ex}
  Find the area of the parallelogram determined by the vectors
  $\begin{mymatrix}{r}
    1 \\
    0 \\
    3
  \end{mymatrix}$, $\begin{mymatrix}{r}
    4 \\
    -2 \\
    1
  \end{mymatrix}$.
  \begin{sol}
    $\begin{mymatrix}{r}
      1 \\
      0 \\
      3
    \end{mymatrix} \times
    \begin{mymatrix}{r}
      4 \\
      -2 \\
      1
    \end{mymatrix} =\begin{mymatrix}{r}
      6 \\
      11 \\
      -2
    \end{mymatrix}$. The area is of the parallelogram is
    $\sqrt{36+121+4}=\sqrt{161}$. 
  \end{sol}
\end{ex}

\begin{ex}
  Find the area of the parallelogram with vertices $(-2,3,1)$,
  $(2,1,1)$, $(1,2,-1)$, and $(5,0,-1)$.
  \begin{sol}
    Let $P=(-2,3,1)$, $Q=(2,1,1)$, $R=(1,2,-1)$, and $S=(5,0,-1)$.
    We have
    $\longvect{PQ}\times\longvect{PR}
    =
    \begin{mymatrix}{r} 4 \\ -2 \\ 0 \end{mymatrix}
    \times
    \begin{mymatrix}{r} 3 \\ -1 \\ -2 \end{mymatrix}
    =
    \begin{mymatrix}{r} 5 \\ 8 \\ 2 \end{mymatrix}$.
    The area of the parallelogram is
    $\norm{\longvect{PQ}\times\longvect{PR}} = \sqrt{5^2+8^2+2^2} =
    \sqrt{93}$.
  \end{sol}
\end{ex}

\begin{ex}
  Find the area of the triangle determined by the three points,
  $(1,0,3),(4,1,0)$ and $(-3,1,1)$.
  \begin{sol}
    Let $P=(1,0,3)$, $Q=(4,1,0)$, and $R=(-3,1,1)$. 
    $\longvect{PQ}\times\longvect{PR}
    = \begin{mymatrix}{r}
      3 \\
      1 \\
      -3
    \end{mymatrix} \times \begin{mymatrix}{r}
      -4 \\
      1 \\
      -2
    \end{mymatrix} =\begin{mymatrix}{c}
      1 \\
      18 \\
      7
    \end{mymatrix}$. The area of the triangle is
    $\displaystyle\frac{1}{2}\norm{\longvect{PQ}\times\longvect{PR}} =
    \frac{1}{2}\sqrt{1+18^2+7^2}=\frac{1}{2}\sqrt{374}$. 
  \end{sol}
\end{ex}

\begin{ex}
  Find the area of the triangle determined by the three points,
  $(1,2,3),(2,3,4)$ and $(3,4,5)$. Did something
  interesting happen here? What does it mean geometrically?
  \begin{sol}
    Let $P=(1,2,3)$, $Q=(2,3,4)$, and $R=(3,4,5)$. 
    $\longvect{PQ}\times\longvect{PR}
    = \begin{mymatrix}{r}
      1 \\ 1 \\ 1
    \end{mymatrix} \times \begin{mymatrix}{r}
      2 \\ 2 \\ 2
    \end{mymatrix} =\begin{mymatrix}{c}
      0 \\ 0 \\ 0
    \end{mymatrix}$.
    The area of the triangle is 0. It means the three points are on
    a line.
  \end{sol}
\end{ex}

