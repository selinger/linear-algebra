\begin{ex}
  Suppose that $A,B,C,D$ are $n\times n$-matrices, and that all
  relevant matrices are invertible. Further, suppose that $(A+B)^{-1}
  = CB^{-1}$. Solve this equation for $A$ (in terms of $B$ and $C$),
  $B$ (in terms of $A$ and $C$), and $C$ (in terms of $A$ and $B$).
  \begin{sol}
    To solve for $A$, we invert both sides of the equation
    $(A+B)^{-1} = CB^{-1}$ and use matrix algebra to get
    $A+B = (CB^{-1})^{-1} = (B^{-1})^{-1}C^{-1} = BC^{-1}$. Therefore,
    $A = BC^{-1} - B$.

    To solve for $B$, we note that $A = BC^{-1} - B = B(C^{-1}-I)$.
    Multiplying both sides of the equation on the right by the inverse
    of $C^{-1}-I$, we get $B = A(C^{-1}-I)^{-1}$.

    To solve for $C$, we take the original equation
    $(A+B)^{-1} = CB^{-1}$ and right-multiply both sides of the
    equation $B$. This yields $C = (A+B)^{-1}B$.
  \end{sol}
\end{ex}
