\section{Definition of vector spaces}

\begin{outcome}
  \begin{enumerate}
  \item Develop the abstract concept of a vector space through axioms.
  \item Deduce basic properties of vector spaces.
  \item Use the vector space axioms to determine if a set and its
    operations constitute a vector space.
  \end{enumerate}
\end{outcome}

\begin{definition}{Vector space}{vector-space}
  Let $F$ be a field. A \textbf{vector space over $F$}\index{vector
    space} is a set $V$ equipped with two operations of
  \textbf{addition}%
  \index{addition!in a vector space}%
  \index{vector!addition} \index{addition!of vectors} and
  \textbf{scalar multiplication}%
  \index{scalar multiplication!in a vector space}%
  \index{vector!scalar multiplication}%
  \index{scalar multiplication!of a vector}%
  \index{multiplication!scalar multiplication|see{scalar multiplication}},
  such that the following properties hold:
  \begin{itemize}\setlength\itemsep{0em}
  \item[(A1)] Commutative law of addition%
    \index{commutative law!of addition}%
    \index{vector!commutative law of addition}:
    $\vect{u} + \vect{v}=\vect{v} + \vect{u}$.
  \item[(A2)] Associative law of addition%
    \index{associative law!of addition}%
    \index{vector!associative law of addition}:
    $(\vect{u} + \vect{v}) + \vect{w}=\vect{u} + (\vect{v} + \vect{w})$.
  \item[(A3)] The existence of an additive unit%
    \index{additive unit}\index{unit!of addition|see{additive unit}}%
    \index{vector!additive unit}: there exists an element $\vect{0}\in
    V$ such that for all $\vect{u}$,
    $\vect{u} + \vect{0}=\vect{u}$.
  \item[(A4)] The law of additive inverses%
    \index{additive inverse}%
    \index{vector!additive inverse}%
    \index{inverse!additive}:
    $\vect{u} + (-\vect{u}) =\vect{0}$.
  \item[(SM1)] The distributive law over vector addition%
    \index{distributive law!over vector addition}%
    \index{vector!distributive law}:
    $k(\vect{u} + \vect{v}) = k\vect{u} + k\vect{v}$.
  \item[(SM2)] The distributive law over scalar addition%
    \index{distributive law!over scalar addition}:
    $(k + \ell) \vect{u} = k \vect{u} + \ell\vect{u}$.
  \item[(SM3)] The associative law for scalar multiplication%
    \index{associative law!of scalar multiplication}%
    \index{vector!associative law of scalar multiplication}:
    $k(\ell\vect{u}) = (k \ell)\vect{u}$.
  \item[(SM4)] The rule for multiplication by one%
    \index{rule for multiplication by 1}%
    \index{vector!rule for multiplication by 1}:
    $1\vect{u}=\vect{u}$.
  \end{itemize}
\end{definition}

The above definition is concerned about two operations: vector
addition, denoted by $\vect{v} + \vect{w}$, and scalar multiplication,
denoted by $k\vect{v}$ or sometimes $k\cdot\vect{v}$. In the law of
additive inverses, we have written $-\vect{u}$ for $(-1)\vect{u}$. Often, the
scalars will be real numbers, but it is also possible to use scalars
from a different field, which is why we have assumed a given field $F$
of scalars in the definition.

Our first example of a vector space is of course $\R^n$.

\begin{proposition}{$\R^n$ is a vector space}{Rn-vector-space}
  The set $\R^n$ of $n$-dimensional column vectors, with the usual
  operations of vector addition and scalar multiplication, is a vector
  space.
\end{proposition}

\begin{proof}
  Properties (A1)--(A4) hold by
  Theorem~\ref{thm:properties-vector-addition}, and properties
  (SM1)--(SM4) hold by Theorem~\ref{thm:vector-scalar-mult}.
\end{proof}

We now consider some other examples of vector spaces.

\begin{example}{Vector space of polynomials}{vector-space-poly}
  Let $\Poly_2$%
  \index{P2@$\Poly_2$}%
  \index{vector space!of polynomials} be the set of all polynomials%
  \index{polynomial} of degree at most $2$, i.e., expressions of the
  form
  \begin{equation*}
    p(x) = ax^2 + bx + c,
  \end{equation*}
  where $a$, $b$, and $c$ are scalars. Define addition%
  \index{polynomial!addition}%
  \index{addition!of polynomials} and scalar multiplication%
  \index{polynomial!scalar multiplication}%
  \index{scalar multiplication!of polynomials} of polynomials in the
  usual way, i.e.,
  \begin{eqnarray*}
    (ax^2 + bx + c) + (a'x^2 + b'x + c') &=& (a + a')x^2 + (b + b')x + (c + c') \\
    k(ax^2 + bx + c) &=& ka\,x^2 + kb\,x + kc.
  \end{eqnarray*}
\end{example}

\begin{proposition}{$\Poly_2$ is a vector space}{vector-space-poly}
  $\Poly_2$ is a vector space.
\end{proposition}

\begin{proof}
  To show that $\Poly_2$ is a vector space, we verify the $8$ vector
  space axioms. Let
  \begin{eqnarray*}
    p(x) &=& a_2x^2 + a_1x + a_0, \\
    q(x) &=& b_2x^2 + b_1x + b_0, \\
    r(x) &=& c_2x^2 + c_1x + c_0 \\
  \end{eqnarray*}
  be polynomials in $\Poly_2$ and let $k,\ell$ be scalars.

  \begin{itemize}
  \item[(A1)] We prove the commutative law of addition.
    \begin{eqnarray*}
      p(x) + q(x)
      &=& (a_2x^2 + a_1x + a_0) + (b_2x^2 + b_1x + b_0) \\
      &=& (a_2 + b_2)x^2 + (a_1 + b_1)x + (a_0 + b_0) \\
      &=& (b_2 + a_2)x^2 + (b_1 + a_1)x + (b_0 + a_0) \\
      &=& (b_2x^2 + b_1x + b_0) + (a_2x^2 + a_1x + a_0) \\
      &=& q(x) + p(x).
    \end{eqnarray*}
  \item[(A2)] We prove the associative law of addition.
    \begin{eqnarray*}
      (p(x) + q(x)) + r(x)
      &=& ((a_2x^2 + a_1x + a_0) + (b_2x^2 + b_1x + b_0)) + (c_2x^2 + c_1x + c_0) \\
      &=& ((a_2 + b_2)x^2 + (a_1 + b_1)x + (a_0 + b_0)) + (c_2x^2 + c_1x + c_0) \\
      &=& ((a_2 + b_2) + c_2)x^2 + ((a_1 + b_1) + c_1)x + ((a_0 + b_0) + c_0) \\
      &=& (a_2 + (b_2 + c_2))x^2 + (a_1 + (b_1 + c_1))x + (a_0 + (b_0 + c_0)) \\
      &=& (a_2x^2 + a_1x + a_0) + ((b_2 + c_2)x^2 + (b_1 + c_1)x + (b_0 + c_0)) \\
      &=& (a_2x^2 + a_1x + a_0) + ((b_2x^2 + b_1x + b_0) + (c_2x^2 + c_1x + c_0)) \\
      &=& p(x) + (q(x) + r(x)).
    \end{eqnarray*}
  \item[(A3)] To prove the existence of an additive unit, let $0(x) =
    0x^2 + 0x + 0$, the so-called \textbf{zero polynomial}%
    \index{polynomial!zero}%
    \index{zero polynomial}. Then
    \begin{eqnarray*}
      p(x) + 0(x)  &=&  (a_2x^2 + a_1x + a_0) + (0x^2 + 0x + 0) \\
                   &=&  (a_2 + 0)x^2 + (a_1 + 0)x + (a_0 + 0)\\
                   &=&  a_2x^2 + a_1x + a_0 \\
                   &=&  p(x).
    \end{eqnarray*}
  \item[(A4)] We prove the law of additive inverses.
    \begin{eqnarray*}
      p(x) + (-p(x)) &=& (a_2x^2 + a_1x + a_0) + (- a_2x^2  - a_1x - a_0) \\
                     &=& (a_2 - a_2)x^2 + (a_1 - a_1)x + (a_0 - a_0) \\
                     &=& 0x^2 + 0x + 0 \\
                     &=& 0(x).
    \end{eqnarray*}
  \item[(SM1)] We prove the distributive law over vector addition.
    \begin{eqnarray*}
      k(p(x) + q(x)) &=& k ((a_2x^2 + a_1x + a_0) + (b_2x^2 + b_1x + b_0)) \\
                     &=& k ((a_2 + b_2)x^2 + (a_1 + b_1)x + (a_0 + b_0)) \\
                     &=& k(a_2 + b_2)x^2 + k(a_1 + b_1)x + k(a_0 + b_0) \\
                     &=& (ka_2 + kb_2)x^2 + (ka_1 + kb_1)x + (ka_0 + kb_0) \\
                     &=& (ka_2x^2 + ka_1x + ka_0) + (kb_2x^2 + kb_1x + kb_0 \\
                     &=& kp(x) + kq(x).
    \end{eqnarray*}
  \item[(SM2)] We prove the distributive law over scalar addition.
    \begin{eqnarray*}
      (k + \ell) p(x) &=& (k + \ell) (a_2x^2 + a_1x + a_0)\\
                 &=& (k + \ell)a_2x^2 + (k + \ell)a_1x + (k + \ell)a_0   \\
                 &=& (ka_2x^2 + ka_1x + ka_0) + (\ell a_2x^2 + \ell a_1x + \ell a_0)\\
                 &=& kp(x) + \ell p(x).
    \end{eqnarray*}
  \item[(SM3)] We prove the associative law for scalar multiplication.
    \begin{eqnarray*}
      k(\ell p(x)) &=& k(\ell(a_2x^2 + a_1x + a_0)) \\
               &=& k(\ell a_2x^2 + \ell a_1x + \ell a_0)\\
               &=&  k\ell a_2x^2 + k\ell a_1x + k\ell a_0 \\
               &=& (k\ell) (a_2x^2 + a_1x + a_0)\\
               &=& (k\ell) p(x).
    \end{eqnarray*}
  \item[(SM4)] Finally, we prove the rule for multiplication by one.
    \begin{eqnarray*}
      1p(x) &=& 1 (a_2x^2 + a_1x + a_0)\\
            &=& 1a_2x^2 + 1a_1x + 1a_0\\
            &=& a_2x^2 + a_1x + a_0\\
            &=& p(x).
    \end{eqnarray*}
  \end{itemize}
  Since the operations of addition and scalar multiplication on
  $\Poly_2$ satisfy the $8$ vector space axioms, $\Poly_2$ is a vector
  space.
\end{proof}

Our next example of a vector space is the set of all
$n\times m$-matrices.

\begin{example}{Vector space of matrices}{vector-space-matrices}
  Let $\Mat_{n,m}$%
  \index{Matnm@$\Mat_{n,m}$} be the set of all $n\times m$-matrices
  with entries in a field $F$%
  \index{vector space!of matrices}%
  \index{matrix!vector space of}, together with the usual operations
  of matrix addition and scalar multiplication.
\end{example}

\begin{proposition}{$\Mat_{n,m}$ is a vector space}{vector-space-matrices}
  $\Mat_{n,m}$ is a vector space.
\end{proposition}

\begin{proof}
  The properties (A1)--(A4) hold by
  Theorem~\ref{thm:properties-of-addition}, and the properties
  (SM1)--(SM4) hold by Theorem~\ref{thm:properties-scalar-mult}.
\end{proof}

We now examine an example of a set that does not satisfy all of the
above axioms, and is therefore \textit{not} a vector space.

\begin{example}{Not a vector space}{not-vector-space}
  Let $V$ denote the set of $2 \times 3$-matrices. Let us define a
  non-standard addition in $V$ by $A \oplus B = A$ for all matrices
  $A,B\in V$. Let scalar multiplication in $V$ be the usual scalar
  multiplication of matrices. Show that $V$ is not a vector space.
\end{example}

\begin{solution}
  In order to show that $V$ is not a vector space, it suffices to find
  one of the 8 axioms that is not satisfied. We will begin by examining
  the axioms for addition until one is found which does not hold. In
  fact, for this example, the very first axiom fails. Let
  \begin{equation*}
    A = \begin{mymatrix}{rrr}
      1 & 0 & 0 \\
      0 & 0 & 0
    \end{mymatrix},
    \quad
    B = \begin{mymatrix}{rrr}
      0 & 0 & 0 \\
      1 & 0 & 0
    \end{mymatrix}.
  \end{equation*}
  Then $A\oplus B=A$ and $B\oplus A=B$. Since $A\neq B$, we have
  $A\oplus B\neq B\oplus A$ for these two matrices, so property (A1)
  is false. 
\end{solution}

Our next example looks a little different. 

\begin{example}{Vector space of functions}{vector-space-function}
  Let $S$ be a nonempty set and define $\Func_S$%
  \index{FuncS@$\Func_S$} to be the set of \textit{real-valued}
  functions%
  \index{function!vector space of} defined on $S$. In other words, the
  elements of $\Func_S$ are functions $f:S\to\R$. The sum of two
  functions is defined by
  \begin{equation*}
    (f + g)(x) = f(x) + g(x),
  \end{equation*}
  and the scalar multiplication is defined by
  \begin{equation*}
    (kf) (x) = k(f(x)).
  \end{equation*}
\end{example}

\begin{proposition}{$\Func_S$ is a vector space}{vector-space-function}
  $\Func_S$ is a vector space.
\end{proposition}

\begin{proof}
  To verify that $\Func_S$ is a vector space, we must prove the 8
  axioms of vector spaces. Let $f, g, h$ be functions in $\Func_S$,
  and let $k,\ell$ be real numbers. Recall that two functions $f,g$
  are \textbf{equal}%
  \index{function!equality of}%
  \index{equality of functions} if for all $x\in S$, we have
  $f(x)=g(x)$.

  \begin{itemize}
  \item[(A1)] We prove the commutative law of addition. For all
    $x\in S$, we have
    \begin{equation*}
      (f + g) (x)
      ~=~ f(x) + g(x)
      ~=~ g(x) + f(x)
      ~=~ (g + f) (x).
    \end{equation*}
    Therefore, $f + g = g + f$.
  \item[(A2)] We prove the associative law of addition. For all
    $x\in S$, we have
    \begin{equation*}
      ((f + g) + h) (x)
      ~=~ (f + g) (x) + h(x)
      ~=~ (f(x) + g(x)) + h(x)
    \end{equation*}
    \begin{equation*}
      ~=~ f(x) + (g(x) + h(x))
      ~=~ (f(x) + (g + h) (x))
      ~=~ (f + (g + h)) (x).
    \end{equation*}
    Therefore, $(f + g) + h = f + (g + h)$.
  \item[(A3)] To prove the existence of an additive unit, let $0$
    denote the function that is given by $0(x)=0$. This is called the
    \textbf{zero function}%
    \index{function!zero function}%
    \index{zero function}. It is an additive unit because for all $x$,
    \begin{equation*}
      (f + 0) (x)
      ~=~ f(x) + 0(x)
      ~=~ f(x),
    \end{equation*}
    and so $f+0 = f$.
  \item[(A4)] We prove the law of additive inverses. Let $-f = (-1)f$
    be the function that satisfies $(-f) (x) = -f(x)$. Then for all $x$,
    \begin{equation*}
      (f + (-f)) (x)
      ~=~ f(x) + (-f) (x)
      ~=~ f(x) + -f(x)
      ~=~ 0.
    \end{equation*}
    Therefore $f + (-f) = 0$.
  \item[(SM1)] We prove the distributive law over vector addition. For
    all $x$, we have
    \begin{equation*}
      ((k + \ell ) f) (x)
      ~=~ (k + \ell ) f(x)
      ~=~ kf(x) + \ell f(x)
      ~=~ (kf + \ell f) (x),
    \end{equation*}
    and so $(k + \ell ) f=kf + \ell f$.
  \item[(SM2)] We prove the distributive law over scalar addition.
    \begin{equation*}
      (k(f + g)) (x)
      ~=~ k(f + g) (x)
      ~=~ k(f(x) + g(x))
    \end{equation*}
    \begin{equation*}
      ~=~ kf(x) + \ell g(x)
      ~=~ (kf + \ell g) (x),
    \end{equation*}
    and so $k(f + g) = kf + \ell g$.
  \item[(SM3)] We prove the associative law for scalar multiplication.
    \begin{equation*}
      ((k\ell ) f) (x)
      ~=~ (k\ell) f(x)
      ~=~ k(\ell f(x))
      ~=~ (k(\ell f)) (x),
    \end{equation*}
    so $(k\ell f) =k(\ell f)$.
  \item[(SM4)] Finally, we prove the rule for multiplication by one.
    For all $x\in S$, we have
    \begin{equation*}
      (1f) ( x) ~=~ 1f(x) ~=~f(x),
    \end{equation*}
    and therefore $1f=f$.
  \end{itemize}

  It follows that $\Func_S$ satisfies all the required axioms and is a
  vector space.
\end{proof}

In the last example, we took $\Func_S$ to be the set of
\textit{real-valued} functions on $S$, i.e., functions whose domain is
the set $S$ and whose values are real numbers. Instead, we could have
also considered the set of functions valued in $F$, where $F$ is an
arbitrary field. Proposition~\ref{prop:vector-space-function} remains
true in this more general case.
