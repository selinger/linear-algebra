\section{Application: Cramer's rule}

Another context in which the formula given in
Theorem~\ref{thm:inverse-and-determinant} is important is
\textbf{Cramer's Rule}.  Recall that we can represent a system of
linear equations in the form $AX=B$, where the solutions to this
system are given by $X$.  Cramer's Rule gives a formula for the
solutions $X$ in the special case that $A$ is a square invertible
matrix. Note this rule does not apply if you have a system of
equations in which there is a different number of equations than
variables (in other words, when $A$ is not square), or when $A$ is not
invertible.

Suppose we have a system of equations given by $AX=B$, and we want to
find solutions $X$ which satisfy this system.  Then recall that if
$A^{-1}$ exists,
\begin{eqnarray*}
  AX&=&B \\
  A^{-1}(AX)&=&A^{-1}B \\
  (A^{-1}A)X&=&A^{-1}B \\
  IX&=&A^{-1}B\\
  X &=& A^{-1}B
\end{eqnarray*}
Hence, the solutions $X$ to the system are given by $X=A^{-1}B$.
Since we assume that $A^{-1}$ exists, we can use the formula for
$A^{-1}$ given above. Substituting this formula into the equation for
$X$, we have
\begin{equation*}
  X=A^{-1}B=\frac{1}{\det(A)}\adj(A)B
\end{equation*}
Let $x_i$ be the $i\th$ entry of $X$ and $b_j$ be the $j\th$ entry of
$B$.  Then this equation becomes
\begin{equation*}
  x_i = \sum_{j=1}^{n}\mat{a_{ij}}^{-1}b_{j}=\sum_{j=1}^{n}\frac{1}{\det(A)} \adj(A)_{ij}b_{j}
\end{equation*}
where $\adj(A)_{ij}$ is the $\ijth$ entry of $\adj(A)$.

By the formula for the expansion of a determinant along a column,
\begin{equation*}
  x_{i}=\frac{1}{\det(A)}\det \begin{mymatrix}{ccccc}
    \ast & \cdots & b_{1} & \cdots & \ast \\
    \vdots &  & \vdots &  & \vdots \\
    \ast & \cdots & b_{n} & \cdots & \ast
  \end{mymatrix} 
\end{equation*}
where here the $i\th$ column of $A$ is replaced with the column vector
$\mat{b_{1},\ldots,b_{n}}^{T}$. The determinant of this modified
matrix is taken and divided by $\det(A)$. This formula is known as
Cramer's rule%
\index{Cramer's rule}.

We formally define this method now.

\begin{procedure}{Using Cramer's rule}{cramers-rule}
  Suppose $A$ is an $n\times n$ invertible matrix and we wish to solve
  the system $AX=B$ for $X =\mat{x_{1},\ldots,x_{n}}^{T}$. Then
  Cramer's rule says
  \begin{equation*}
    x_{i}=
    \frac{\det(A_{i})}{\det(A)}
  \end{equation*}
  where $A_{i}$ is the matrix obtained by replacing the $i\th$ column
  of $A$ with the column matrix
  \begin{equation*}
    B = 
    \begin{mymatrix}{c}
      b_1 \\
      \vdots \\
      b_n
    \end{mymatrix}
  \end{equation*} 
\end{procedure}

We illustrate this procedure in the following example.

\begin{example}{Using Cramer's rule}{cramers-rule}
  Find $x,y,z$ if
  \begin{equation*}
    \begin{mymatrix}{rrr}
      1 & 2 & 1 \\
      3 & 2 & 1 \\
      2 & -3 & 2
    \end{mymatrix} \begin{mymatrix}{c}
      x \\
      y \\
      z
    \end{mymatrix} =\begin{mymatrix}{r}
      1 \\
      2 \\
      3
    \end{mymatrix} 
  \end{equation*}
\end{example}

\begin{solution}
  We will use method outlined in Procedure~\ref{proc:cramers-rule} to
  find the values for $x,y,z$ which give the solution to this system.
  Let
  \begin{equation*}
    B = 
    \begin{mymatrix}{r}
      1 \\
      2 \\
      3
    \end{mymatrix} 
  \end{equation*}
  In order to find $x$, we calculate
  \begin{equation*}
    x =
    \frac{\det(A_{1})}{\det(A)}
  \end{equation*}
  where $A_1$ is the matrix obtained from replacing the first column
  of $A$ with $B$.

  Hence, $A_1$ is given by 
  \begin{equation*}
    A_1 = 
    \begin{mymatrix}{rrr}
      1 & 2 & 1 \\
      2 & 2 & 1 \\
      3 & -3 & 2
    \end{mymatrix}
  \end{equation*}
  Therefore,
  \begin{equation*}
    x=
    \frac{\det(A_{1})}{\det(A)}
    =
    \frac{\begin{absmatrix}{rrr}
        1 &  2 & 1 \\
        2 &  2 & 1 \\
        3 & -3 & 2
      \end{absmatrix} }{\begin{absmatrix}{rrr}
        1 & 2 & 1 \\
        3 & 2 & 1 \\
        2 & -3 & 2
      \end{absmatrix} }=\frac{1}{2}
  \end{equation*}
  Similarly, to find $y$ we construct $A_2$ by replacing the second
  column of $A$ with $B$. Hence, $A_2$ is given by
  \begin{equation*}
    A_2
    =
    \begin{mymatrix}{rrr}
      1 & 1 & 1 \\
      3 & 2 & 1 \\
      2 & 3 & 2
    \end{mymatrix}
  \end{equation*}
  Therefore,
  \begin{equation*}
    y=\frac{\det(A_{2})}{\det(A)}
    = \frac{\begin{absmatrix}{rrr}
        1 & 1 & 1 \\
        3 & 2 & 1 \\
        2 & 3 & 2
      \end{absmatrix} }{\begin{absmatrix}{rrr}
        1 & 2 & 1 \\
        3 & 2 & 1 \\
        2 & -3 & 2
      \end{absmatrix} }
    = -\frac{1}{7}
  \end{equation*}
  Similarly, $A_3$ is constructed by replacing the third column of $A$
  with $B$. Then, $A_3$ is given by
  \begin{equation*}
    A_3
    =
    \begin{mymatrix}{rrr}
      1 & 2 & 1 \\
      3 & 2 & 2 \\
      2 & -3 & 3
    \end{mymatrix}
  \end{equation*}
  Therefore, $z$ is calculated as follows.
  \begin{equation*}
    z=
    \frac{\det(A_{3})}{\det(A)}
    =
    \frac{\begin{absmatrix}{rrr}
        1 & 2 & 1 \\
        3 & 2 & 2 \\
        2 & -3 & 3
      \end{absmatrix} }{\begin{absmatrix}{rrr}
        1 & 2 & 1 \\
        3 & 2 & 1 \\
        2 & -3 & 2
      \end{absmatrix} }=\frac{11}{14}
  \end{equation*}
\end{solution}

Cramer's Rule gives you another tool to consider when solving a system
of linear equations.

We can also use Cramer's Rule for systems of non-linear
equations. Consider the following system where the matrix $A$ has
functions rather than numbers for entries.

\begin{example}{Use Cramer's rule for non-constant matrix}{cramers-rule-non-constant-matrix}
  Solve for $z$ if
  \begin{equation*}
    \begin{mymatrix}{ccc}
      1 & 0 & 0 \\
      0 & e^{t}\cos t & e^{t}\sin t \\
      0 & -e^{t}\sin t & e^{t}\cos t
    \end{mymatrix} \begin{mymatrix}{c}
      x \\
      y \\
      z
    \end{mymatrix} =\begin{mymatrix}{c}
      1 \\
      t \\
      t^{2}
    \end{mymatrix}
  \end{equation*}
\end{example}

\begin{solution}
  We are asked to find the value of $z$ in the solution. We will solve
  using Cramer's rule.  Thus
  \begin{equation*}
    z=\vspace{.05in} \frac{\begin{absmatrix}{ccc}
        1 & 0 & 1 \\
        0 & e^{t}\cos t & t \\
        0 & -e^{t}\sin t & t^{2}
      \end{absmatrix} }{\begin{absmatrix}{ccc}
        1 & 0 & 0 \\
        0 & e^{t}\cos t & e^{t}\sin t \\
        0 & -e^{t}\sin t & e^{t}\cos t
      \end{absmatrix} }=\allowbreak t((\cos t) t+\sin t) e^{-t}
  \end{equation*}
\end{solution}
