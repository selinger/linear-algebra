\section{Application: A formula for the inverse of a matrix}
\label{sec:adjugate}

\begin{outcome}
  \begin{enumerate}
  \item Find the cofactor matrix and the adjugate of a matrix.
  \item Find the inverse of a matrix using the adjugate formula.
  \end{enumerate}
\end{outcome}

The determinant of a matrix also provides a way to find the inverse of
a matrix.  Recall the definition of the inverse of a matrix from
Definition~\ref{def:invertible-matrix}. If $A$ is an
$n\times n$-matrix, we say that $A^{-1}$ is the inverse of $A$ if
$AA^{-1} = I$ and $A^{-1}A=I$.

We now define a new matrix called the \textbf{cofactor matrix} of $A$.
The cofactor matrix of $A$ is the matrix whose $\ijth$ entry is the
$\ijth$ cofactor of $A$.

\begin{definition}{The cofactor matrix}{cofactor-matrix}
  Let $A$ be an $n\times n$-matrix. Then the \textbf{cofactor matrix
    of $A$}%
  \index{cofactor matrix}%
  \index{matrix!cofactor matrix}, denoted $\cof(A)$, is defined by
  \begin{equation*}
    \cof(A)
    ~=~ \mat{\cofactor{A}{ij}}
    ~=~ \begin{mymatrix}{cccc}
      \cofactor{A}{11} & \cofactor{A}{12} & \cdots & \cofactor{A}{1n} \\
      \cofactor{A}{21} & \cofactor{A}{22} & \cdots & \cofactor{A}{2n} \\
      \vdots & \vdots & \ddots & \vdots \\
      \cofactor{A}{n1} & \cofactor{A}{n2} & \cdots & \cofactor{A}{nn} \\
    \end{mymatrix},
  \end{equation*}
  where $\cofactor{A}{ij}$ is the $\ijth$ cofactor of $A$.
\end{definition}

We will use the cofactor matrix to create a formula for the inverse of
$A$. First, we define the \textbf{adjugate}%
\index{adjugate of a matrix}%
\index{matrix!adjugate} of $A$, denoted $\adj(A)$, to be the transpose
of the cofactor matrix:
\begin{equation*}
  \adj(A) = \cof(A)^T.
\end{equation*}
The adjugate is also sometimes called the
\textbf{classical adjoint}%
\index{classical adjoint}%
\index{matrix!classical adjoint} of $A$.

\begin{example}{Cofactor matrix and adjugate}{cofactor-matrix-and-adjugate}
  Find the cofactor matrix and the adjugate of $A$, where
  \begin{equation*}
    A = \begin{mymatrix}{rrr}
      1 & 2 & 3 \\
      3 & 0 & 1 \\
      1 & 2 & 1 \\
    \end{mymatrix}.
  \end{equation*}
\end{example}

\begin{solution}
  We first find $\cof(A)$. To do so, we need to compute the cofactors
  of $A$. We have:
  \begin{eqnarray*}
    \cofactor{A}{11} ~=~ +\minor{A}{11}
    &=&
    \begin{absmatrix}{ccc}
      \strikeh{3.2em}{\strikev{3.2em}{1}} & 2 & 3 \\
      3 & 0 & 1 \\
      1 & 2 & 1 \\
    \end{absmatrix}
    ~=~ \begin{absmatrix}{rr}
      0 & 1 \\
      2 & 1 \\
    \end{absmatrix}
    ~=~ -2,
    \\
    \cofactor{A}{12} ~=~ -\minor{A}{12}
    &=&
    -\begin{absmatrix}{ccc}
      \strikeh{3.2em}{1} & \strikev{3.2em}{2} & 3 \\
      3 & 0 & 1 \\
      1 & 2 & 1 \\
    \end{absmatrix}
    ~=~ -\begin{absmatrix}{rr}
      3 & 1 \\
      1 & 1 \\
    \end{absmatrix}
    ~=~ -2,
    \\
    \cofactor{A}{13} ~=~ +\minor{A}{13}
    &=&
    \begin{absmatrix}{ccc}
      \strikeh{3.2em}{1} & 2 & \strikev{3.2em}{3} \\
      3 & 0 & 1 \\
      1 & 2 & 1 \\
    \end{absmatrix}
    ~=~ \begin{absmatrix}{rr}
      3 & 0 \\
      1 & 2 \\
    \end{absmatrix}
    ~=~ 6,
    \\
    \cofactor{A}{21} ~=~ -\minor{A}{21}
    &=&
    -\begin{absmatrix}{ccc}
      \strikev{3.2em}{1} & 2 & 3 \\
      \strikeh{3.2em}{3} & 0 & 1 \\
      1 & 2 & 1 \\
    \end{absmatrix}
    ~=~ -\begin{absmatrix}{rr}
      2 & 3 \\
      2 & 1 \\
    \end{absmatrix}
    ~=~ 4,
  \end{eqnarray*}
  and so on. Continuing in this way, we find the cofactor matrix
  \begin{equation*}
    \cof(A)
    =
    \begin{mymatrix}{rrr}
      -2 & -2 & 6 \\
      4 & -2 & 0 \\
      2 & 8 & -6 \\
    \end{mymatrix}.
  \end{equation*}
  Finally, the adjugate is the transpose of the cofactor matrix:
  \begin{equation*}
    \adj(A) = \cof(A)^T =
    \begin{mymatrix}{rrr}
      -2 & 4 & 2 \\
      -2 & -2 & 8 \\
      6 & 0 & -6 \\
    \end{mymatrix}.
  \end{equation*}
\end{solution}

The following theorem provides a formula for $A^{-1}$ using the
determinant and the adjugate of $A$.

\begin{theorem}{Formula for the inverse}{inverse-and-determinant}
  Let $A$ be an $n\times n$-matrix. Then
  \begin{equation*}
    A \, \adj(A) = \adj(A)\,A = \det(A)\,I.
  \end{equation*}
  Moreover, $A$ is invertible if and only if $\det(A) \neq 0$. In
  this case, we have:
  \begin{equation*}
    A^{-1} = \frac{1}{\det(A)} \adj(A).
  \end{equation*}
  We call this the \textbf{adjugate formula}%
  \index{determinant!formula for matrix inverse}%
  \index{adjugate formula}%
  \index{inverse!of a matrix!adjugate formula}%
  \index{matrix!inverse!adjugate formula} for the matrix inverse.
\end{theorem}

\begin{proof}
  Recall that the $(i,j)$-entry of $\adj(A)$ is equal to
  $\cofactor{A}{ji}$.  Thus the $(i,j)$-entry of $B=A\,\adj(A)$ is:
  \begin{eqnarray*}
    B_{ij}
    &=& a_{i1}\adj(A)_{1j} + a_{i2}\adj(A)_{2j} + \ldots + a_{in}\adj(A)_{nj} \\
    &=& a_{i1}\cofactor{A}{j1} + a_{i2}\cofactor{A}{j2} + \ldots + a_{in}\cofactor{A}{jn}.
  \end{eqnarray*}
  By the cofactor expansion theorem, we see that this expression for
  $B_{ij}$ is equal to the determinant of the matrix obtained from $A$
  by replacing its $j$th row by $\mat{a_{i1}, a_{i2}, \dots a_{in}}$,
  i.e., by its $i$th row.

  If $i=j$ then this matrix is $A$ itself and therefore
  $B_{ii}=\det(A)$. If on the other hand $i\neq j$, then this matrix
  has its $i$th row equal to its $j$th row, and therefore $B_{ij}=0$
  in this case. Thus we obtain:
  \begin{equation*}
    A \, \adj(A) = {\det(A)} I.
  \end{equation*}
  By a similar argument (using columns instead of rows), we can verify that:
  \begin{equation*}
    \adj(A)\,A = {\det(A)} I.
  \end{equation*}
  This proves the first part of the theorem. For the second part,
  assume that $A$ is invertible. Then by
  Theorem~\ref{thm:determinant-invertible}, $\det(A)\neq 0$. Dividing the
  formula from the first part of the theorem by $\det(A)$, we obtain
  \begin{equation*}
    A\paren{\frac{1}{\det(A)}\adj(A)} ~=~ \paren{\frac{1}{\det(A)}\adj(A)}A ~=~ I,
  \end{equation*}
  and therefore
  \begin{equation*}
    A^{-1} = \frac{1}{\det(A)} \adj(A).
  \end{equation*}
  This completes the proof.
\end{proof}

\begin{example}{Finding the inverse using a formula}{inverse-and-determinant}
  Use the adjugate formula to find the inverse of the matrix
  \begin{equation*}
    A=\begin{mymatrix}{rrr}
      1 & 2 & 3 \\
      3 & 0 & 1 \\
      1 & 2 & 1 \\
    \end{mymatrix}.
  \end{equation*}
\end{example}

\begin{solution}
  We must compute
  \begin{equation*}
    A^{-1} ~=~ \frac{1}{\det(A)} \adj(A).
  \end{equation*}
  We will start by computing the determinant. We expand along the
  second row:
  \begin{equation*}
    \det(A)
    ~=~ \begin{absmatrix}{rrr}
      1 & 2 & 3 \\
      3 & 0 & 1 \\
      1 & 2 & 1 \\
    \end{absmatrix}
    ~=~ -3 \begin{absmatrix}{rr}
      2 & 3 \\
      2 & 1 \\
    \end{absmatrix}
    - 1 \begin{absmatrix}{rr}
      1 & 2 \\
      1 & 2 \\
    \end{absmatrix}
    ~=~ -3(-4) -1(0) ~=~ 12.
  \end{equation*}
  We have already calculated the adjugate $\adj(A)$ in
  Example~\ref{exa:cofactor-matrix-and-adjugate}:
  \begin{equation*}
    \adj(A) ~=~
    \begin{mymatrix}{rrr}
      -2 & 4 & 2 \\
      -2 & -2 & 8 \\
      6 & 0 & -6 \\
    \end{mymatrix}.
  \end{equation*}
  Therefore, the inverse of $A$ is given by
  \begin{equation*}
    \def\arraystretch{1.4}
    A^{-1}
    ~=~ \frac{1}{\det(A)}\adj(A)
    ~=~ \frac{1}{12}\begin{mymatrix}{rrr}
      -2 & 4 & 2 \\
      -2 & -2 & 8 \\
      6 & 0 & -6 \\
    \end{mymatrix}
    ~=~ \begin{mymatrix}{rrr}
      -\frac{1}{6} & \frac{1}{3} & \frac{1}{6} \\
      -\frac{1}{6} & -\frac{1}{6} & \frac{2}{3} \\
      \frac{1}{2} & 0 & -\frac{1}{2} \\
    \end{mymatrix}.
  \end{equation*}
  Since it is very easy to make a mistake in this calculation, we
  double-check our answer by computing $A^{-1}A$:
  \begin{equation*}
    \def\arraystretch{1.4}
    A^{-1}A ~=~
    \allowbreak \begin{mymatrix}{rrr}
      -\frac{1}{6} & \frac{1}{3} &
      \frac{1}{6} \\
      -\frac{1}{6} & -\frac{1}{6} &
      \frac{2}{3} \\
      \frac{1}{2} & 0 & -\frac{1}{2}
    \end{mymatrix} \begin{mymatrix}{rrr}
      1 & 2 & 3 \\
      3 & 0 & 1 \\
      1 & 2 & 1
    \end{mymatrix} ~=~ \begin{mymatrix}{rrr}
      1 & 0 & 0 \\
      0 & 1 & 0 \\
      0 & 0 & 1
    \end{mymatrix}
    ~=~
    I.
  \end{equation*}
\end{solution}

\begin{example}{Finding the inverse using a formula}{inverse-formula}
  Use the adjugate formula to find the inverse of the matrix
  \begin{equation*}
    A=\begin{mymatrix}{rrr}
      0  &  2 &  1 \\
      -1 &  2 &  2 \\
      2  & -2 & -2 \\
    \end{mymatrix}.
  \end{equation*}
\end{example}

\begin{solution}
  We start by calculating the determinant:
  \begin{equation*}
    \det(A)
    ~=~ \begin{absmatrix}{rrr}
      0  &  2 &  1 \\
      -1 &  2 &  2 \\
      2  & -2 & -2 \\
    \end{absmatrix}
    ~=~ -2 \begin{absmatrix}{rr}
      -1 &  2 \\
      2  & -2 \\
    \end{absmatrix}
    + 1 \begin{absmatrix}{rrr}
      -1 &  2 \\
      2  & -2 \\
    \end{absmatrix}
    ~=~ -2\cdot (-2) + 1\cdot(-2) ~=~ 2.
  \end{equation*}
  Next, we compute the cofactor matrix:
  \begin{equation*}
    \cof(A)
    ~=~
    \begin{mymatrix}{ccc}
      ~~~\begin{absmatrix}{rr}
       2 &  2 \\
       -2 & -2 \\
      \end{absmatrix}
      &
      -\begin{absmatrix}{rr}
      -1 &  2 \\
      2  & -2 \\
      \end{absmatrix}
      &
      ~~~\begin{absmatrix}{rr}
      -1 &  2 \\
      2  & -2 \\
      \end{absmatrix}
      \\\\[-1ex]
      -\begin{absmatrix}{rr}
      2 &  1 \\
      -2 & -2 \\
      \end{absmatrix}
      &
      ~~~\begin{absmatrix}{rr}
      0 &  1 \\
      2 & -2 \\
      \end{absmatrix}
      &
      -\begin{absmatrix}{rr}
      0  &  2 \\
      2  & -2 \\
      \end{absmatrix}
      \\\\[-1ex]
      ~~~\begin{absmatrix}{rr}
      2 &  1 \\
      2 &  2 \\
      \end{absmatrix}
      &
      -\begin{absmatrix}{rr}
      0  &  1 \\
      -1 &  2 \\
      \end{absmatrix}
      &
      ~~~\begin{absmatrix}{rr}
      0  &  2 \\
      -1 &  2 \\
      \end{absmatrix}
    \end{mymatrix}
    ~=~ \begin{mymatrix}{rrr}
      0 &  2 & -2 \\
      2 & -2 &  4 \\
      2 & -1 &  2 \\
    \end{mymatrix}.
  \end{equation*}
  The adjugate is the transpose of the cofactor matrix:
  \begin{equation*}
    \adj(A) ~=~ \cof(A)^T
    ~=~ \begin{mymatrix}{rrr}
      0  &  2 &  2 \\
      2  & -2 & -1 \\
      -2 &  4 &  2 \\
    \end{mymatrix}.
  \end{equation*}
  We therefore have
  \begin{equation*}
    \def\arraystretch{1.2}
    A^{-1}
    ~=~
    \frac{1}{\det(A)}\adj(A)
    ~=~
    \frac{1}{2}
    \begin{mymatrix}{rrr}
      0  &  2 &  2 \\
      2  & -2 & -1 \\
      -2 &  4 &  2 \\
    \end{mymatrix}
    ~=~
    \begin{mymatrix}{rrr}
      0  &  1 &  1 \\
      1  & -1 & -\frac{1}{2} \\
      -1 &  2 &  1 \\
    \end{mymatrix}.
  \end{equation*}
  Once again, we double-check our work by computing $A^{-1}A$:
  \begin{equation*}
    \def\arraystretch{1.2}
    A^{-1}A ~=~
    \begin{mymatrix}{rrr}
      0  &  1 &  1 \\
      1  & -1 & -\frac{1}{2} \\
      -1 &  2 &  1 \\
    \end{mymatrix}
    \begin{mymatrix}{rrr}
      0  &  2 &  1 \\
      -1 &  2 &  2 \\
      2  & -2 & -2 \\
    \end{mymatrix}
    ~=~ \begin{mymatrix}{rrr}
      1 & 0 & 0 \\
      0 & 1 & 0 \\
      0 & 0 & 1
    \end{mymatrix}.
  \end{equation*}
\end{solution}

It is always a good idea to double-check your work.  At the end of the
calculation, it is very easy to compute $A^{-1}A$ and check whether it
is equal to $I$. If they are not equal, be sure to go back and
double-check each step. One common mistake is to forget to take the
transpose of the cofactor matrix, so be sure not to forget this step.

In practice, it is usually much faster to compute the inverse by the
method of Section~\ref{ssec:computing-inverses}, because this only
requires solving a single system of equations, rather than computing a
large number of cofactors. However, there are some situations where
the adjugate formula is useful. One such situation is when the matrix
has complicated entries that are functions rather than numbers. The
following example illustrates this.

\begin{example}{Inverse for non-constant matrix}{inverse-non-constant-matrix}
  Let
  \begin{equation*}
    A(t) =\begin{mymatrix}{ccc}
      e^{t} & 0 & 0 \\
      0 & \cos t & \sin t \\
      0 & -\sin t & \cos t
    \end{mymatrix}.
  \end{equation*}
  Show that $A(t)^{-1}$ exists and find it.
\end{example}

\begin{solution}
  First note that
  \begin{equation*}
    \det(A(t)) ~=~ e^{t}(\cos^2 t + \sin^2 t) ~=~ e^{t}\neq 0.
  \end{equation*}
  Therefore $A(t)^{-1}$ exists for all values of the variable $t$. The
  cofactor matrix is
  \begin{equation*}
    \cof(A(t))
    ~=~ \begin{mymatrix}{ccc}
      1 & 0 & 0 \\
      0 & e^{t}\cos t & e^{t}\sin t \\
      0 & -e^{t}\sin t & e^{t}\cos t
    \end{mymatrix}.
  \end{equation*}
  The adjugate is the transpose of the cofactor matrix, and therefore
  the inverse is
  \begin{equation*}
    A(t)^{-1}
    ~=~ \frac{1}{\det(A(t))}\adj(A(t))
    ~=~
    \frac{1}{e^{t}}\begin{mymatrix}{ccc}
      1 & 0 & 0 \\
      0 & e^{t}\cos t & -e^{t}\sin t \\
      0 & e^{t}\sin t & e^{t}\cos t
    \end{mymatrix}
    ~=~ \begin{mymatrix}{ccc}
      e^{-t} & 0 & 0 \\
      0 & \cos t & -\sin t \\
      0 & \sin t & \cos t
    \end{mymatrix}.
  \end{equation*}
\end{solution}

Another situation where the adjugate formula is useful is the case of a
$2\times 2$-matrix. In this case both the determinant and the adjugate
are especially easy to compute. For a $2\times 2$-matrix
\begin{equation*}
  A ~=~ \begin{mymatrix}{rr}
    a & b \\
    c & d
  \end{mymatrix},
\end{equation*}
we have
\begin{eqnarray*}
  \det(A) &=& ad-bc
  \\
  \adj(A) &=&
  \begin{mymatrix}{rr}
    d & -b \\
    -c & a
  \end{mymatrix}.
\end{eqnarray*}
Therefore, $A$ is invertible if and only if $ad-bc\neq 0$, and in
that case, the inverse is given by
\begin{equation}\label{eqn:inverse-2-by-2}
  A^{-1} ~=~
  \frac{1}{ad-bc}
  \begin{mymatrix}{rr}
    d & -b \\
    -c & a
  \end{mymatrix}.
\end{equation}

\begin{example}{Inverse of a $2\times 2$-matrix}{inverse-2-by-2}
  Find the inverse of
  \begin{equation*}
    A ~=~ \begin{mymatrix}{rr}
      7 & 5 \\
      2 & 2 \\
    \end{mymatrix}.
  \end{equation*}
\end{example}

\begin{solution}
  We use formula {\eqref{eqn:inverse-2-by-2}} to compute the inverse:
  \begin{equation*}
    \def\arraystretch{1.4}
    A^{-1} ~=~
    \frac{1}{7\cdot 2-2\cdot 5}
    \begin{mymatrix}{rr}
      2 & -5 \\
      -2 & 7
    \end{mymatrix}
    ~=~
    \begin{mymatrix}{rr}
      \frac{1}{2} & -\frac{5}{4} \\
      -\frac{1}{2} & \frac{7}{4}
    \end{mymatrix}.
  \end{equation*}
\end{solution}
