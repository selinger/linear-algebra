\section{Determinants and row operations}

Recall that there are three kinds of elementary row operations%
\index{matrix!row operation}%
\index{matrix!elementary row operation}%
\index{row operation}%
\index{elementary row operation} on matrices:
\begin{enumerate}
\item Switch two rows.
\item Multiply a row by a non-zero number.
\item Add a multiple of one row to another row.
\end{enumerate}
The following theorem examines the effect of these row operations on
the determinant of a matrix.

\begin{theorem}{Effect of row operations on the determinant}{determinant-row-operations}
  Let $A$ be an $n\times n$-matrix.
  \begin{enumerate}
  \item If $B$ is obtained from $A$ by switching two rows, then
    \begin{equation*}
      \det(B) = -\det(A).
    \end{equation*}
  \item If $B$ is obtained from $A$ by multiplying one row by a
    non-zero scalar $k$, then
    \begin{equation*}
      \det(B) = k\det(A).
    \end{equation*}
  \item If $B$ is obtained from $A$ by adding a multiple of one row to
    another row, then
    \begin{equation*}
      \det(B) = \det(A).
    \end{equation*}
  \end{enumerate}
\end{theorem}

Notice that the second part of this theorem is true when we multiply
{\em one} row of the matrix by $k$.  If we were to multiply {\em two}
rows of $A$ by $k$ to obtain $B$, we would have
$\det(B) = k^2 \det(A)$.

\begin{example}{Using row operations to calculate a determinant}{determinant-row-operations1}
  Use row operations to calculate the following determinant:
  \begin{equation*}
    \begin{absmatrix}{rrr}
      1 & 5 & 5 \\
      0 & 0 & -3 \\
      0 & 2 & 7 \\
    \end{absmatrix}.
  \end{equation*}
\end{example}

\begin{solution}
  If we switch the second and third rows, we obtain a triangular
  matrix, of which the determinant is easy to compute. By
  Theorem~\ref{thm:determinant-row-operations}, switching two rows
  negates the determinant. We therefore have:
  \begin{equation*}
    \begin{absmatrix}{rrr}
      1 & 5 & 5 \\
      0 & 0 & -3 \\
      0 & 2 & 7 \\
    \end{absmatrix}
    ~=~
    -\begin{absmatrix}{rrr}
      1 & 5 & 5 \\
      0 & 2 & 7 \\
      0 & 0 & -3 \\
    \end{absmatrix}
    ~=~ -(1\cdot 2\cdot(-3)) ~=~ 6.
  \end{equation*}
\end{solution}

\begin{example}{Using row operations to calculate a determinant}{determinant-row-operations2}
  Use row operations to calculate the following determinant:
  \begin{equation*}
    \begin{absmatrix}{rrr}
      1 & 4 & -2 \\
      1 & 8 & 1 \\
      2 & 4 & -9 \\
    \end{absmatrix}.
  \end{equation*}
\end{example}

\begin{solution}
  We can use elementary row operations to reduce this matrix to
  triangular form:
  \begin{equation*}
    \begin{mymatrix}{rrr}
      1 & 4 & -2 \\
      1 & 8 & 1 \\
      2 & 4 & -9 \\
    \end{mymatrix}
    \stackrel{R_2\rowop R_2-R_1}{\sim}
    \begin{mymatrix}{rrr}
      1 & 4 & -2 \\
      0 & 4 & 3 \\
      2 & 4 & -9 \\
    \end{mymatrix}
    \stackrel{R_3\rowop R_3-2R_1}{\sim}
    \begin{mymatrix}{rrr}
      1 & 4 & -2 \\
      0 & 4 & 3 \\
      0 & -4 & -5 \\
    \end{mymatrix}
    \stackrel{R_3\rowop R_3+R_2}{\sim}
    \begin{mymatrix}{rrr}
      1 & 4 & -2 \\
      0 & 4 & 3 \\
      0 & 0 & -2 \\
    \end{mymatrix}.
  \end{equation*}
  Each of the row operations is of the form ``add a multiple of one
  row to another row'', and therefore does not change the
  determinant. We therefore have:
  \begin{equation*}
    \begin{absmatrix}{rrr}
      1 & 4 & -2 \\
      1 & 8 & 1 \\
      2 & 4 & -9 \\
    \end{absmatrix}
    ~=~
    \begin{absmatrix}{rrr}
      1 & 4 & -2 \\
      0 & 4 & 3 \\
      2 & 4 & -9 \\
    \end{absmatrix}
    ~=~
    \begin{absmatrix}{rrr}
      1 & 4 & -2 \\
      0 & 4 & 3 \\
      0 & -4 & -5 \\
    \end{absmatrix}
    ~=~
    \begin{absmatrix}{rrr}
      1 & 4 & -2 \\
      0 & 4 & 3 \\
      0 & 0 & -2 \\
    \end{absmatrix}
    ~=~ 1\cdot 4\cdot(-2) = -8.
  \end{equation*}
\end{solution}

In general, we can convert any square matrix to triangular form using
elementary row operations. In fact, it is always possible to do so
using only elementary operations of the first and third kind (swap two
rows or add a multiple of one row to another). This gives us a very
efficient way to compute determinants. If the matrices are large, this
method is much more efficient than the cofactor method.

\begin{example}{Using row operations to calculate a determinant}{determinant-row-operations3}
  Use elementary row operations of the first and third kind to calculate the
  following determinant:
  \begin{equation*}
    \begin{absmatrix}{rrrr}
      0 & 2 & 1 & 4 \\
      2 & 2 & -4 & -1 \\
      1 & 1 & -2 & -1 \\
      1 & 3 & 2 & 5 \\
    \end{absmatrix}.
  \end{equation*}
\end{example}

\begin{solution}
  We use elementary row operations to reduce the matrix to triangular
  form:
  \begin{equation*}
    \begin{array}{ccccc}
      \begin{mymatrix}{rrrr}
        0 & 2 & 1 & 4 \\
        2 & 2 & -4 & -1 \\
        1 & 1 & -2 & -1 \\
        1 & 3 & 2 & 5 \\
      \end{mymatrix}
      &\stackrel{R_1\rowswap R_3}{\sim}&
      \begin{mymatrix}{rrrr}
        1 & 1 & -2 & -1 \\
        2 & 2 & -4 & -1 \\
        0 & 2 & 1 & 4 \\
        1 & 3 & 2 & 5 \\
      \end{mymatrix}
      &\stackrel{R_2\rowop R_2-2R_1}{\sim}&
      \begin{mymatrix}{rrrr}
        1 & 1 & -2 & -1 \\
        0 & 0 & 0 & 1 \\
        0 & 2 & 1 & 4 \\
        1 & 3 & 2 & 5 \\
      \end{mymatrix}
      \\\\[-1ex]
      &\stackrel{R_4\rowop R_4-R_1}{\sim}&
      \begin{mymatrix}{rrrr}
        1 & 1 & -2 & -1 \\
        0 & 0 & 0 & 1 \\
        0 & 2 & 1 & 4 \\
        0 & 2 & 4 & 6 \\
      \end{mymatrix}
      &\stackrel{R_4\rowop R_4-R_3}{\sim}&
      \begin{mymatrix}{rrrr}
        1 & 1 & -2 & -1 \\
        0 & 0 & 0 & 1 \\
        0 & 2 & 1 & 4 \\
        0 & 0 & 3 & 2 \\
      \end{mymatrix}
      \\\\[-1ex]
      &\stackrel{R_2\rowswap R_3}{\sim}&
      \begin{mymatrix}{rrrr}
        1 & 1 & -2 & -1 \\
        0 & 2 & 1 & 4 \\
        0 & 0 & 0 & 1 \\
        0 & 0 & 3 & 2 \\
      \end{mymatrix}
      &\stackrel{R_3\rowswap R_4}{\sim}&
      \begin{mymatrix}{rrrr}
        1 & 1 & -2 & -1 \\
        0 & 2 & 1 & 4 \\
        0 & 0 & 3 & 2 \\
        0 & 0 & 0 & 1 \\
      \end{mymatrix}
    \end{array}
  \end{equation*}
  By Theorem~\ref{thm:determinant-row-operations}, the determinant
  changes signs each time we swap two rows. The determinant is
  unchanged when we add a multiple of one row to another. Therefore,
  we have
  \begin{equation*}
    \begin{array}{ccccccc}
      \begin{absmatrix}{rrrr}
        0 & 2 & 1 & 4 \\
        2 & 2 & -4 & -1 \\
        1 & 1 & -2 & -1 \\
        1 & 3 & 2 & 5 \\
      \end{absmatrix}
      &=&
      -\begin{absmatrix}{rrrr}
        1 & 1 & -2 & -1 \\
        2 & 2 & -4 & -1 \\
        0 & 2 & 1 & 4 \\
        1 & 3 & 2 & 5 \\
      \end{absmatrix}
      &=&
      -\begin{absmatrix}{rrrr}
        1 & 1 & -2 & -1 \\
        0 & 0 & 0 & 1 \\
        0 & 2 & 1 & 4 \\
        1 & 3 & 2 & 5 \\
      \end{absmatrix}
      \\\\[-1ex]
      &=&
      -\begin{absmatrix}{rrrr}
        1 & 1 & -2 & -1 \\
        0 & 0 & 0 & 1 \\
        0 & 2 & 1 & 4 \\
        0 & 2 & 4 & 6 \\
      \end{absmatrix}
      &=&
      -\begin{absmatrix}{rrrr}
        1 & 1 & -2 & -1 \\
        0 & 0 & 0 & 1 \\
        0 & 2 & 1 & 4 \\
        0 & 0 & 3 & 2 \\
      \end{absmatrix}
      \\\\[-1ex]
      &=&
      +\begin{absmatrix}{rrrr}
        1 & 1 & -2 & -1 \\
        0 & 2 & 1 & 4 \\
        0 & 0 & 0 & 1 \\
        0 & 0 & 3 & 2 \\
      \end{absmatrix}
      &=&
      -\begin{absmatrix}{rrrr}
        1 & 1 & -2 & -1 \\
        0 & 2 & 1 & 4 \\
        0 & 0 & 3 & 2 \\
        0 & 0 & 0 & 1 \\
      \end{absmatrix}
      &=& -6.
    \end{array}
  \end{equation*}
  In practice, the last calculation could have been done in a single
  step. All we had to do is count the number of swap operations we
  performed during the row operations. If there is an odd number of
  swap operations, the sign of the determinant changes; otherwise, it
  stays the same.
\end{solution}

% ======================================================================
\subsection{CONTINUE HERE...}

Suppose we were to multiply all $n$ rows of
$A$ by $k$ to obtain the matrix $B$, so that $B = kA$. Then,
$\det(B) = k^n \det(A)$. This gives the next theorem.

\begin{theorem}{Scalar multiplication}{scalar-mult-determinant}
  Let $A$ and $B$ be $n \times n$-matrices and $k$ a scalar, such that
  $B = kA$. Then $\det(B) = k^n \det(A)$.
\end{theorem}

Consider the following example.

\begin{example}{Multiplying a row by 5}{5-times-row}
  Let $A=\begin{mymatrix}{rr}
    1 & 2 \\
    3 & 4
  \end{mymatrix} ,\ B=\begin{mymatrix}{rr}
    5 & 10 \\
    3 & 4
  \end{mymatrix}$.
  Knowing that $\det(A) =-2$, find  $\det(B)$.
\end{example}

\begin{solution}
  By Definition~\ref{def:two-by-two-determinant}, $\det(A) =-2$. We
  can also compute $\det(B)$ using
  Definition~\ref{def:two-by-two-determinant}, and we see that
  $\det(B) = -10$.

  Now, let's compute $\det(B)$ using
  Theorem~\ref{thm:multiplying-row-by-scalar} and see if we obtain the
  same answer. Notice that the first row of $B$ is $5$ times the first
  row of $A$, while the second row of $B$ is equal to the second row
  of $A$.  By Theorem~\ref{thm:multiplying-row-by-scalar},
  $\det(B) = 5 \times \det(A) = 5 \times -2 = -10$.

  You can see that this matches our answer above.
\end{solution}

Finally, consider the next theorem for the last row operation, that of
adding a multiple of a row to another row.

\begin{theorem}{Adding a multiple of a row to another row}{adding-multiple-of-row}
  Let $A$ be an $n\times n$-matrix and let $B$ be a matrix which
  results from adding a multiple of a row to another row.  Then
  $\det(A) =\det(B)$.
\end{theorem}

Therefore, when we add a multiple of a row to another row, the
determinant of the matrix is unchanged.  Note that if a matrix $A$
contains a row which is a multiple of another row, $\det(A)$ will
equal $0$. To see this, suppose the first row of $A$ is equal to $-1$
times the second row. By Theorem~\ref{thm:adding-multiple-of-row}, we
can add the first row to the second row, and the determinant will be
unchanged. However, this row operation will result in a row of zeros.
Using Laplace Expansion along the row of zeros, we find that the
determinant is $0$.

Consider the following example.

\begin{example}{Adding a row to another row}{adding-row}
  Let $A=\begin{mymatrix}{rr}
    1 & 2 \\
    3 & 4
  \end{mymatrix} $ and let $B=\begin{mymatrix}{rr}
    1 & 2 \\
    5 & 8
  \end{mymatrix}$.
  Find $\det(B)$.
\end{example}

\begin{solution}
  By Definition~\ref{def:two-by-two-determinant}, $\det(A) = -2$.
  Notice that the second row of $B$ is two times the first row of $A$
  added to the second row.  By Theorem~\ref{thm:switching-rows},
  $\det(B) = \det(A) =-2$.  As usual, you can verify this answer
  using Definition~\ref{def:two-by-two-determinant}.
\end{solution}

\begin{example}{Multiple of a row}{multiple-rows}
  Let $A = \begin{mymatrix}{rr}
    1 & 2 \\
    2 & 4
  \end{mymatrix}$. Show that $\det(A) = 0$.
\end{example}

\begin{solution}
  Using Definition~\ref{def:two-by-two-determinant}, the determinant
  is given by
  \begin{equation*}
    \det(A) = 1 \times 4 - 2 \times 2 = 0
  \end{equation*}

  However notice that the second row is equal to $2$ times the first
  row. Then by the discussion above following
  Theorem~\ref{thm:adding-multiple-of-row} the determinant will equal
  $0$.
\end{solution}

Until now, our focus has primarily been on row operations. However, we
can carry out the same operations with columns, rather than rows. The
three operations outlined in Definition~\ref{def:operations} can be
done with columns instead of rows.  In this case, in
Theorems~\ref{thm:switching-rows},
{\ref{thm:multiplying-row-by-scalar}}, and
{\ref{thm:adding-multiple-of-row}} you can replace the word, "row"
with the word "column".

There are several other major properties of determinants which do not
involve row (or column) operations. The first is the determinant of a
product of matrices.

\begin{theorem}{Determinant of a product}{determinant-of-product}
  Let $A$ and $B$ be two $n\times n$-matrices. Then\index{determinant!product}
  \begin{equation*}
    \det(AB) =\det(A) \det(B)
  \end{equation*}
\end{theorem}

In order to find the determinant of a product of matrices, we can
simply take the product of the determinants.

Consider the following example.

\begin{example}{The determinant of a product}{determinant-of-product}
  Compare $\det(AB) $ and $\det(A) \det(
  B) $ for
  \begin{equation*}
    A=\begin{mymatrix}{rr}
      1 & 2 \\
      -3 & 2
    \end{mymatrix} ,B=\begin{mymatrix}{rr}
      3 & 2 \\
      4 & 1
    \end{mymatrix}
  \end{equation*}
\end{example}

\begin{solution} First compute $AB$, which is given by
  \begin{equation*}
    AB=\begin{mymatrix}{rr}
      1 & 2 \\
      -3 & 2
    \end{mymatrix} \begin{mymatrix}{rr}
      3 & 2 \\
      4 & 1
    \end{mymatrix} = \begin{mymatrix}{rr}
      11 & 4 \\
      -1 & -4
    \end{mymatrix}
  \end{equation*}
  and so by Definition~\ref{def:two-by-two-determinant}
  \begin{equation*}
    \det(AB) =\det\begin{mymatrix}{rr}
      11 & 4 \\
      -1 & -4
    \end{mymatrix} = -40
  \end{equation*}
  Now
  \begin{equation*}
    \det(A) =\det\begin{mymatrix}{rr}
      1 & 2 \\
      -3 & 2
    \end{mymatrix} = 8
  \end{equation*}
  and
  \begin{equation*}
    \det(B) =\det\begin{mymatrix}{rr}
      3 & 2 \\
      4 & 1
    \end{mymatrix} = -5
  \end{equation*}
  Computing $\det(A) \times \det(B)$ we have $8 \times -5 =
  -40$. This is the same answer as above and you can see that
  $\det(A) \det(B) =8\times (-5) =-40 = \det(AB)$.
\end{solution}

Consider the next important property.

\begin{theorem}{Determinant of the transpose}{transpose-determinant}
  Let $A$ be a matrix where $A^T$ is the transpose of $A$. Then,
  \begin{equation*}
    \det(A^T) = \det(A)
  \end{equation*}
\end{theorem}

This theorem is illustrated in the following example.

\begin{example}{Determinant of the transpose}{transpose-determinant}
  Let
  \begin{equation*}
    A
    =
    \begin{mymatrix}{rr}
      2 & 5 \\
      4 & 3
    \end{mymatrix}
  \end{equation*}
  Find $\det(A^T)$.
\end{example}

\begin{solution}
  First, note that
  \begin{equation*}
    A^{T}
    =
    \begin{mymatrix}{rr}
      2 & 4 \\
      5 & 3
    \end{mymatrix}
  \end{equation*}

  Using Definition~\ref{def:two-by-two-determinant}, we can compute
  $\det(A)$ and $\det(A^T)$. It follows that
  $\det(A) = 2 \times 3 - 4 \times 5 = -14$ and
  $\det(A^T) = 2 \times 3 - 5 \times 4 = -14$.  Hence,
  $\det(A) = \det(A^T)$.
\end{solution}

The following provides an essential property of the determinant, as
well as a useful way to determine if a matrix is invertible.

\begin{theorem}{Determinant of the inverse}{determinant-inverse}
  Let $A$ be an $n \times n$-matrix. Then $A$ is invertible if and
  only if $\det(A) \neq 0$. If this is true, it follows that
  \begin{equation*}
    \det(A^{-1}) = \frac{1}{\det(A)}
  \end{equation*}
\end{theorem}

Consider the following example.

\begin{example}{Determinant of an invertible matrix}{determinant-invertible-matrix}
  Let $A = \begin{mymatrix}{rr}
    3 & 6 \\
    2 & 4
  \end{mymatrix}, B = \begin{mymatrix}{rr}
    2 & 3 \\
    5 & 1
  \end{mymatrix}$. For each matrix, determine if it is invertible. If
  so, find the determinant of the inverse.
\end{example}

\begin{solution}
  Consider the matrix $A$ first. Using
  Definition~\ref{def:two-by-two-determinant} we can find the
  determinant as follows:
  \begin{equation*}
    \det(A) = 3 \times 4 - 2 \times 6 = 12 - 12 = 0
  \end{equation*}
  By Theorem~\ref{thm:determinant-inverse} $A$ is not invertible.

  Now consider the matrix $B$. Again by
  Definition~\ref{def:two-by-two-determinant} we have
  \begin{equation*}
    \det(B) = 2 \times 1 - 5 \times 3 = 2 - 15 = -13
  \end{equation*}
  By Theorem~\ref{thm:determinant-inverse} $B$ is invertible and the
  determinant of the inverse is given by
  \begin{eqnarray*}
    \det(A^{-1}) &=& \frac{1}{\det(A)} \\
                  &=& \frac{1}{-13} \\
                  &=& -\frac{1}{13}
  \end{eqnarray*}
\end{solution}
