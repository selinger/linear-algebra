\section{Matrix arithmetic}

\begin{outcome}
  \begin{enumerate}
  \item Perform the matrix operations of matrix addition, scalar
    multiplication, transposition and matrix multiplication. Identify
    when these operations are not defined. Represent these operations in
    terms of the entries of a matrix.
  \item Prove algebraic properties for matrix addition, scalar
    multiplication, transposition, and matrix multiplication. Apply these
    properties to manipulate an algebraic expression involving matrices.
  \item Compute the inverse of a matrix using row operations, and prove
    identities involving matrix inverses.
  \item Solve a linear system using matrix algebra.
  \item Use multiplication by an elementary matrix to apply row operations.
  \item Write a matrix as a product of elementary matrices.
  \end{enumerate}
\end{outcome}

We have solved systems of equations by writing them in terms of an
augmented matrix and then doing row operations. It turns out that
matrices are important not only for systems of equations but also for
many other purposes.

\begin{definition}{Matrix}{matrix}
  A \textbf{matrix}\index{matrix} is a rectangular array of numbers
  \begin{equation*}
    A = \begin{mymatrix}{cccc}
      a_{11} & a_{12} & \cdots & a_{1n} \\
      a_{21} & a_{22} & \cdots & a_{2n} \\
      \vdots & \vdots & \ddots & \vdots \\
      a_{m1} & a_{m2} & \cdots & a_{mn} \\
    \end{mymatrix},
  \end{equation*}
  where the $a_{ij}$ are scalars, called the
  \textbf{entries}\index{matrix!entry of}%
  \index{entry of a matrix} or
  \textbf{components}\index{matrix!component of}%
  \index{component!of a matrix} of $A$.  The
  \textbf{size}\index{matrix!size of}\index{size of a matrix} or
  \textbf{dimension}\index{matrix!dimension of}%
  \index{dimension!of a matrix} of a matrix is defined as $m\times n$,
  where $m$ is the number of rows and $n$ is the number of columns.
\end{definition}

For example, here is a $3\times 4$-matrix (pronounced ``three-by-four
matrix''):
\begin{equation*}
  \begin{mymatrix}{rrrr}
    1 & 2 & 3 & 4 \\
    5 & 2 & 8 & 7 \\
    6 & -9 & 1 & 2
  \end{mymatrix}.
\end{equation*}
This is a $3\times 4$ matrix because there are three rows and four
columns. When specifying the size of a matrix, we always list the
number of rows before the number of columns.

Entries of the matrix are identified according to their position. The
\textbf{$(i,j)$-entry}\index{matrix!ij-entry@$(i,j)$-entry}%
\index{ij-entry of a matrix@$(i,j)$-entry of a matrix} of a matrix is
the entry in the $i\th$ row and $j\th$ column, and is often denoted
$a_{ij}$. For example, in the above matrix, the $(2,3)$-entry is equal
to $8$, because it is in the second row and the third column.  We
sometimes use $A=\mat{a_{ij}}$ as a short-hand notation for the entire
$m\times n$-matrix whose $(i,j)$-entry is equal to $a_{ij}$ for all
$i=1,\ldots m$ and $j=1,\ldots,n$.

\begin{definition}{Square matrix}{square-matrix}
  A matrix $A$ of size $n\times n$ is called a \textbf{square
    matrix}\index{matrix!square}\index{square matrix}.  In other
  words, $A$ is a square matrix if it has the same number of rows and
  columns.
\end{definition}

% ======================================================================
% CONTINUE HERE

There are various operations which are done on matrices of appropriate
sizes. Matrices can be added to and subtracted from other matrices,
multiplied by a scalar, and multiplied by other matrices. We will
never divide a matrix by another matrix, but we will see later how
matrix inverses play a similar role.

In doing arithmetic with matrices, we often define the action by what
happens in terms of the entries (or components) of the
matrices. Before looking at these operations in depth, consider a few
general definitions.

\begin{definition}{The zero matrix}{zero-matrix}
The \textbf{$m\times n$ zero matrix} is the $m\times n$ matrix
having every entry equal to zero. It is\index{zero matrix} denoted by $0.$
\end{definition}

One possible zero matrix is shown in the following example.

\begin{example}{The zero matrix}{zero-matrix}
The $2\times 3$ zero matrix is $0= \begin{mymatrix}{ccc}
0 & 0 & 0 \\
0 & 0 & 0
\end{mymatrix} $.
\end{example}

Note there is a $2\times 3$ zero matrix, a $3\times 4$ zero matrix, etc. In
fact there is a zero matrix for every size!

\begin{definition}{Equality of matrices}{equality-of-matrices}
 Let $A$ and $B$ be two $m \times n$ matrices. Then $A=B$\index{matrix!equality} means
that for $A=\mat{a_{ij}} $
and $B=\mat{b_{ij}} ,$ $a_{ij}=b_{ij}$ for all $1\leq i\leq m$ and
$1\leq j\leq n.$
\end{definition}

In other words, two matrices are equal exactly when they are the same size and the
corresponding entries are identical. Thus
\begin{equation*}
\begin{mymatrix}{rr}
0 & 0 \\
0 & 0 \\
0 & 0
\end{mymatrix} \neq \begin{mymatrix}{rr}
0 & 0 \\
0 & 0
\end{mymatrix}
\end{equation*}
because they are different sizes.
Also,
\begin{equation*}
\begin{mymatrix}{rr}
0 & 1 \\
3 & 2
\end{mymatrix} \neq \begin{mymatrix}{rr}
1 & 0 \\
2 & 3
\end{mymatrix}
\end{equation*}
because, although they are the same size, their corresponding entries are not identical.

In the following section, we explore addition of matrices.
