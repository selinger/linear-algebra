\section{Definition and examples}

\begin{outcome}
  \begin{enumerate}
  \item Understand the definition of a linear transformation in the
    context of vector spaces.
  \item Determine whether a function is linear or not.
  \end{enumerate}
\end{outcome}

In Chapter~\ref{cha:linear-transformation}, we defined a linear
transformation $T:\R^n\to\R^m$ to be a function that preserves
addition and scalar multiplication. We now revisit this concept in the
more general setting of vector spaces $V$ and $W$.

\begin{definition}{Linear transformation}{linear-transformation-vector-space}
  Let $V$ and $W$ be vector spaces over some field $K$. A function
  $T: V \to W$ is called a \textbf{linear transformation}%
  \index{linear transformation}%
  \index{linear transformation!on vector spaces} from $V$ to $W$ if
  it satisfies the following two conditions:
  \begin{enumerate}
  \item $T$ preserves addition, i.e., for all\/
    $\vect{v},\vect{w}\in V$, we have
    $T(\vect{v}+\vect{w}) = T(\vect{v}) + T(\vect{w})$;
  \item $T$ preserves scalar multiplication, i.e, for all\/
    $\vect{v}\in V$ and $k\in K$, we have
    $T(k\vect{v}) = kT(\vect{v})$.
  \end{enumerate}
  A linear transformation is also sometimes called a \textbf{linear
    function}%
  \index{linear function|see{linear transformation}} or a
  \textbf{linear map}%
  \index{linear map|see{linear transformation}}. A linear
  transformation $T:V\to V$ (i.e., when $V=W$) is also sometimes
  called an \textbf{operator}%
  \index{operator|seealso{linear transformation}}.
\end{definition}

Our first example of a linear transformation is a matrix
transformation. We have already seen this in
Section~\ref{sec:matrix-of-transformation}.

\begin{example}{Matrix transformation}{matrix-transformation}
  Let $A$ be an $m\times n$-matrix. Then the function $T:\R^m\to\R^n$
  defined by $T(\vect{v})=A\vect{v}$ is a linear transformation,
  called a \textbf{matrix transformation}%
  \index{matrix transformation}. This was proved in
  Proposition~\ref{prop:matrix-are-linear}.
\end{example}

There are many interesting examples of linear transformations on
vector spaces other than $\R^n$. We will consider a few such examples.

\begin{example}{Derivative operator}{derivative-operator}
  Let $\Poly_n$ be the vector space of real polynomials of degree at
  most $n$. The function $D:\Poly_n\to\Poly_{n-1}$ is defined by
  \begin{equation*}
    D(p(x)) = p'(x),
  \end{equation*}
  where $p'(x)$ denotes the derivative of the polynomial $p(x)$. The
  function $D$ is called the \textbf{derivative operator}.
  \begin{enumialphparenastyle}
    \begin{enumerate}
    \item Compute $D(x^3)$, $D(2x^2+x)$, and $D(ax^3+bx^2+cx+d)$.
    \item Show that $D$ is a linear transformation.
    \end{enumerate}
  \end{enumialphparenastyle}
\end{example}

\begin{solution}
  \begin{enumialphparenastyle}
    \begin{enumerate}
    \item In each case, we simply take the derivative:
      \begin{eqnarray*}
        D(x^3) &=& 3x^2, \\
        D(2x^2+x) &=& 4x+1, \\
        D(ax^3+bx^2+cx+d) &=& 3ax^2 + 2bx + c.
      \end{eqnarray*}
    \item First, we note that if $p(x)$ is a polynomial of degree at
      most $n$, then its derivative $p'(x)$ is a polynomial of degree
      at most $n-1$. Therefore, the derivative operator $D$ is a
      well-defined function from $\Poly_n$ to $\Poly_{n-1}$. To show that
      it preserves addition, consider any two polynomials
      $p(x),q(x)\in\Poly_n$. From calculus, we know that the
      derivative of $p(x)+q(x)$ is $p'(x)+q'(x)$. Therefore,
      \begin{equation*}
        D(p(x)+q(x)) ~=~ (p(x)+q(x))' ~=~ p'(x)+q'(x) ~=~ D(p(x)) + D(q(x)),
      \end{equation*}
      and so $D$ preserves addition. To show that it preserves scalar
      multiplication, consider $p(x)\in\Poly_n$ and $k\in\R$. From
      calculus, we know that the derivative of $kp(x)$ is $kp'(x)$, and
      therefore
      \begin{equation*}
        D(kp(x)) ~=~ (kp(x))' ~=~ kp'(x) ~=~ kD(p(x)).
      \end{equation*}
      Hence, $D$ preserves scalar multiplication. It follows that $D$ is a
      linear transformation.
    \end{enumerate}
  \end{enumialphparenastyle}
\end{solution}

It is important to understand that we are not claiming that the
derivative $p'(x)$ of a polynomial $p(x)$ is a linear function. It is
of course a polynomial. Rather, what the above example shows is that
the act of {\em taking} the derivative is a linear operation, i.e.,
the derivative of a sum is the sum of the derivatives, and the
derivative of a constant times a function is a constant times the
derivative.

\begin{example}{A differential equation}{differential-equation-derivative-operator}
  Solve the equation $p(x) = x^3 + D(p(x))$, where
  $D:\Poly_3\to\Poly_2$ is the derivative operator of
  Example~\ref{exa:derivative-operator}.
\end{example}

\begin{solution}
  Every element of $P_3$ is of the form $p(x) = ax^3 + bx^2 + cx + d$.
  We can write the equation $p(x) = x^3 + D(p(x))$ as
  \begin{equation*}
    (ax^3 + bx^2 + cx + d) = x^3 + (3ax^2 + 2bx + c).
  \end{equation*}
  For the left-hand side and right-hand side to be equal, we must have
  $a=1$, $b=3a$, $c=2b$, and $d=c$. This yields the unique solution
  $(a,b,c,d) = (1,3,6,6)$, or $p(x) = x^3+3x^2+6x+6$.
\end{solution}

\begin{example}{The shift and unshift operators}{shift-unshift}
  Consider the vector space $\Seq_K$ of infinite sequences of elements
  of $K$. The function $\shift:\Seq_K\to\Seq_K$ is defined by shifting
  the entire sequence to the left and dropping the first element:
  \begin{equation*}
    \shift(a_0,a_1,a_2,a_3,\ldots) = (a_1,a_2,a_3,a_4,\ldots).
  \end{equation*}
  The function $\unshift:\Seq_K\to\Seq_K$ is defined by shifting the
  entire sequence to the right and adding $0$ as the new first
  element:
  \begin{equation*}
    \unshift(a_0,a_1,a_2,a_3,\ldots) = (0,a_0,a_1,a_2,\ldots).
  \end{equation*}
  \begin{enumialphparenastyle}
    \begin{enumerate}
    \item Compute $\shift(1,2,3,\ldots)$,
      $\unshift(\shift(1,1,1,\ldots))$, and
      $\shift(\unshift(1,1,1,\ldots))$.
    \item Show that $\shift$ and $\unshift$ are linear transformations.
    \end{enumerate}
  \end{enumialphparenastyle}
\end{example}

\begin{solution}
  \begin{enumialphparenastyle}
    \begin{enumerate}
    \item We have
      \begin{eqnarray*}
        \shift(1,2,3,4,\ldots)
        &=& (2,3,4,5,\ldots), \\
        \unshift(\shift(1,1,1,1,\ldots))
        &=& \unshift(1,1,1,1,\ldots)
        ~~=~~ (0,1,1,1,\ldots), \\
        \shift(\unshift(1,1,1,1,\ldots))
        &=& \shift(0,1,1,1,\ldots)
        ~~=~~ (1,1,1,1,\ldots).
      \end{eqnarray*}
    \item To show that $\shift$ is a linear transformation, we show
      that it preserves addition and scalar multiplication. Let
      $a=(a_0,a_1,a_2,\ldots)$ and $b=(b_0,b_1,b_2,\ldots)$. Then
      \begin{eqnarray*}
        \shift(a+b) &=& \shift(a_0+b_0,~a_1+b_1,~a_2+b_2,~\ldots) \\
                    &=& (a_1+b_1,~a_2+b_2,~a_3+b_3,~\ldots) \\
                    &=& (a_1,~a_2,~a_3,~\ldots) + (b_1,~b_2,~b_3,~\ldots) \\
                    &=& \shift(a) + \shift(b),~ \\
        \shift(ka)  &=& \shift(ka_0,~ka_1,~ka_2,~\ldots) \\
                    &=& (ka_1,~ka_2,~ka_3,~\ldots) \\
                    &=& k(a_1,~a_2,~a_3,~\ldots) \\
                    &=& k\shift(a).
      \end{eqnarray*}
      Therefore, $\shift$ is a linear transformation. The proof for
      $\unshift$ is similar.
    \end{enumerate}
  \end{enumialphparenastyle}
\end{solution}

\begin{example}{Properties of shift and unshift}{shift-unshift-properties}
  Show that for all sequences $a\in\Seq_K$, we have
  $\shift(\unshift(a))=a$. On the other hand, show that in general,
  $\unshift(\shift(a))\neq a$.
\end{example}

\begin{solution}
  For $a=(a_0,a_1,a_2,\ldots)$, we have
  \begin{equation*}
    \shift(\unshift(a))
    ~=~ \shift(\unshift(a_0,a_1,a_2,\ldots))
    ~=~ \shift(0,a_0,a_1,\ldots)
    ~=~ (a_0,a_1,a_2,\ldots)
    ~=~ a.
  \end{equation*}
  On the other hand, we have
  \begin{equation*}
    \unshift(\shift(a))
    ~=~ \unshift(\shift(a_0,a_1,a_2,\ldots))
    ~=~ \unshift(a_1,a_2,a_3,\ldots)
    ~=~ (0,a_1,a_2,\ldots).
  \end{equation*}
  The latter is not equal to $a$ unless $a_0=0$.
\end{solution}

\begin{example}{Recurrence as a linear equation}{recurrence-as-equation}
  Find the general solution to the following equation, where
  $a\in\Seq_K$:
  \begin{equation*}
    \shift(\shift(a)) = \shift(a) + a.
  \end{equation*}
  Where have you seen this equation before?
\end{example}

\begin{solution}
  For $a=(a_0,a_1,a_2,\ldots)$, we have
  \begin{eqnarray*}
    \shift(\shift(a)) &=& (a_2,a_3,a_4,\ldots), \\
    \shift(a) &=& (a_1,a_2,a_3,\ldots), \\
    a &=& (a_0,a_1,a_2,\ldots).
  \end{eqnarray*}
  Therefore $a$ is a solution of the equation $\shift(\shift(a)) =
  \shift(a) + a$ if and only if
  \begin{eqnarray*}
    a_2 &=& a_1 + a_0, \\
    a_3 &=& a_2 + a_1, \\
    a_4 &=& a_3 + a_2,
  \end{eqnarray*}
  and so on. In other words, $a$ is a solution if and only if
  $a_{n+2} = a_{n+1} + a_n$ holds for all $n\geq 0$. This is nothing
  but the recurrence relation of
  Example~\ref{exa:subspace-recurrence}. We already calculated the
  general solution in
  Example~{\ref{exa:subspace-recurrence-dimension}}. The general
  solution is
  \begin{equation*}
    a = (x,~y,~x+y,~x+2y,~2x+3y,~3x+5y,~\ldots),
  \end{equation*}
  and a basis for the solution space is
  \begin{equation*}
    \set{(1,0,1,1,2,3,\ldots),~ (0,1,1,2,3,5,\ldots)}.
  \end{equation*}
\end{solution}

We conclude this section by stating some elementary properties of
linear transformations. ``Elementary'' means that these properties
follow directly from the definition, i.e., from the fact that linear
transformations preserve addition and scalar multiplication.

\begin{theorem}{Properties of linear transformations}{properties}
  Let $V$ and $W$ be vector spaces over a field $K$, and let
  $T:V \to W$ be a linear transformation.  Then
  \begin{itemize}
  \item $T$ preserves the zero vector:
    $T(\vect{0})=\vect{0}$.
  \item $T$ preserves additive inverses:
    $T(-\vect{v})= -T(\vect{v})$.
  \item $T$ preserves linear combinations:
    \begin{equation*}
      T(a_1\vect{v}_1 + a_2\vect{v}_2 + \ldots + a_k\vect{v}_k)
      ~=~
      a_1T(\vect{v}_1) + a_2T(\vect{v}_2) + \ldots + a_kT(\vect{v}_k).
    \end{equation*}
  \end{itemize}
\end{theorem}

\begin{proof}
  To prove the first property, let $k=0$ in the equation
  $T(k\vect{v}) = kT(\vect{v})$. Since $0\vect{v}=\vect{0}$ and
  $0T(\vect{v})=\vect{0}$ by
  Proposition~\ref{prop:vector-space-elementary}, we therefore have
  $T(\vect{0}) = \vect{0}$.  Similarly, to prove the second property,
  let $k=-1$ in the equation $T(k\vect{v}) = kT(\vect{v})$.
  Finally, the third property is a direct consequence of the fact that
  $T$ preserves addition and scalar multiplication:
  \begin{eqnarray*}
    T(a_1\vect{v}_1 + a_2\vect{v}_2 + \ldots + a_k\vect{v}_k)
    &=& T(a_1\vect{v}_1) + T(a_2\vect{v}_2) + \ldots + T(a_k\vect{v}_k) \\
    &=& a_1T(\vect{v}_1) + a_2T(\vect{v}_2) + \ldots + a_kT(\vect{v}_k).
  \end{eqnarray*}
\end{proof}
