\section{Properties of eigenvectors and eigenvalues}

\begin{theorem}{Linearly independent eigenvectors}{linearly-independent-eigenvectors}
  Let $A$ be an $n\times n$-matrix, and suppose that $A$ has distinct
  eigenvalues $\eigenvar_1, \eigenvar_2, \ldots, \eigenvar_m$.  For each
  $i$, let $X_i$ be a $\eigenvar_i$-eigenvector of $A$.  Then
  $\set{X_1, X_2, \ldots, X_m}$ is linearly independent.
\end{theorem}

The corollary that follows from this theorem gives a useful tool in
determining if $A$ is diagonalizable.

\begin{corollary}{Distinct eigenvalues}{distinct-eigenvalues}
  Let $A$ be an $n \times n$-matrix and suppose it has $n$ distinct
  eigenvalues. Then it follows that $A$ is diagonalizable.
\end{corollary}

The idea that a matrix may not be diagonalizable suggests that
conditions exist to determine when it is possible to diagonalize a
matrix. We saw earlier in Corollary~\ref{cor:distinct-eigenvalues}
that an $n \times n$-matrix with $n$ distinct eigenvalues is
diagonalizable. It turns out that there are other useful
diagonalizability tests.

Recall that the multiplicity of an eigenvalue $\eigenvar$ is the number
of times that it occurs as a root of the characteristic polynomial.

Consider now the following lemma.

\begin{lemma}{Dimension of the eigenspace}{dimension-eigenspace}
  If $A$ is an $n\times n$-matrix, then
  \begin{equation*}
    \dim(E_{\eigenvar}(A))\leq m,
  \end{equation*}
  where $\eigenvar$ is an eigenvalue of $A$ of multiplicity $m$.
\end{lemma}

This result tells us that if $\eigenvar$ is an eigenvalue of $A$, then
the number of linearly independent $\eigenvar$-eigenvectors is never
more than the multiplicity of $\eigenvar$. We now use this fact to
provide a useful diagonalizability condition.

\begin{theorem}{Diagonalizability condition}{diagonalizability}
  Let $A$ be an $n \times n$-matrix $A$. Then $A$ is diagonalizable if
  and only if for each eigenvalue $\eigenvar$ of $A$,
  $\dim(E_{\eigenvar}(A))$ is equal to the multiplicity of $\eigenvar$.
\end{theorem}
