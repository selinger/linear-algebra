\section{Finding eigenvalues}

In the previous section, we saw how to find the eigenvectors
corresponding to a given eigenvalue $\lambda$, if $\lambda$ is already
known. But we have not yet seen how to find the eigenvalues of a
matrix. However, the calculations in
Examples~\ref{exa:find-eigenvectors-given-eigenvalue} and
{\ref{exa:basis-eigenspace}} suggest a way forward. We can see that
the following are equivalent:
\begin{enumerate}
\item $\lambda$ is an eigenvalue of $A$.
\item There exists a non-zero vector $\vect{v}$ such that
  $A\vect{v}=\lambda\vect{v}$.
\item The homogeneous system of equations
  $(A-\lambda I)\vect{v}=\vect{0}$ has a non-trivial solution.
\end{enumerate}
Indeed, the equivalence between 1 and 2 is just the definition of an
eigenvalue, and the equivalence between 2 and 3 is just algebra.
By Corollary~\ref{cor:determinant-homogeneous}, we know that the
system $(A-\lambda I)\vect{v}=\vect{0}$ has a non-trivial solution if
and only if $\det(A-\lambda I)=0$. Therefore, we have proved the
following theorem:

\begin{theorem}{Eigenvalues}{eigenvalues}
  Let $A$ be a square matrix, and let $\lambda$ be a scalar. Then
  $\lambda$ is an eigenvalue of $A$ if and only if
  \begin{equation*}
    \det(A-\lambda I)=0.
  \end{equation*}
\end{theorem}

\begin{example}{Finding the eigenvalues}{finding-eigenvalues}
  Find the eigenvalues of the matrix
  \begin{equation*}
    A = \begin{mymatrix}{rr}
      -5 & 2 \\
      -7 & 4 \\
    \end{mymatrix}.
  \end{equation*}
\end{example}

\begin{solution}
  By Theorem~\ref{thm:eigenvalues}, a scalar $\lambda$ is an
  eigenvalue of $A$ if and only if $\det(A-\lambda I)=0$. We calculate
  the determinant:
  \begin{eqnarray*}
    \det(A-\lambda I)
    &=&
    \begin{absmatrix}{cc}
      -5-\lambda & 2 \\
      -7 & 4-\lambda \\
    \end{absmatrix} \\
    &=& (-5-\lambda)(4-\lambda) + 14 \\
    &=& \lambda^2 + \lambda - 6.
  \end{eqnarray*}
  Therefore, $\lambda$ is an eigenvalue if and only if
  $\lambda^2 + \lambda - 6 = 0$. We can find the roots of this
  equation using the quadratic formula, or equivalently, by factoring
  the left-hand side:
  \begin{equation*}
    \lambda^2 + \lambda - 6 = 0
    \iff
    (\lambda+3)(\lambda-2) = 0.
  \end{equation*}
  Therefore, the eigenvalues are $\lambda=-3$ and $\lambda=2$.
\end{solution}

\begin{example}{Finding the eigenvalues}{finding-eigenvalues2}
  Find the eigenvalues of the matrix
  \begin{equation*}
    A=\begin{mymatrix}{rrr}
      5 & -4 & 4 \\
      2 & -1 & 2 \\
      0 &  0 & 2 \\
    \end{mymatrix}.
  \end{equation*}
\end{example}

\begin{solution}
  Once again, we calculate $\det(A-\lambda I)$:
  \begin{eqnarray*}
    \det(A-\lambda I)
    &=&
        \begin{absmatrix}{ccc}
          5-\lambda & -4 & 4 \\
          2 & -1-\lambda & 2 \\
          0 &  0 & 2-\lambda \\
        \end{absmatrix} \\
    &=&
        (5-\lambda)(-1-\lambda)(2-\lambda) - 2(-4)(2-\lambda) \\
    &=& -\lambda^3 + 6\lambda^2 - 11\lambda + 6 \\
    &=& (3-\lambda)(1-\lambda)(2-\lambda).
  \end{eqnarray*}
  The eigenvalues are the roots of this polynomial, i.e., the
  solutions of the equation
  $(\lambda-3)(\lambda-1)(2-\lambda)=0$. Therefore, the eigenvalues of
  $A$ are $\lambda=1$, $\lambda=2$, and $\lambda=3$.
\end{solution}

As the examples show, the quantity $\det(A-\lambda I)$ is always a
polynomial in the variable $\lambda$. A \textbf{polynomial}%
\index{polynomial} is an expression of the form
\begin{equation*}
  p(\lambda) = a_n\lambda^n + a_{n-1}\lambda^{n-1} + \ldots + a_1\lambda + a_0,
\end{equation*}
where $a_0,\ldots,a_n$ are constants called the \textbf{coefficients}
of the polynomial. The polynomial $\det(A-\lambda I)$ has a special
name:

\begin{definition}{Characteristic polynomial}{characteristic-polynomial}
  Let $A$ be a square matrix. The expression
  \begin{equation*}
    p(\lambda) = \det(A-\lambda I)
  \end{equation*}
  is called the \textbf{characteristic polynomial}%
  \index{characteristic polynomial}%
  \index{polynomial!characteristic}%
  \index{matrix!characteristic polynomial} of $A$.
\end{definition}

\begin{example}{Characteristic polynomial}{characteristic-polynomial}
  Find the characteristic polynomial of the matrix
  \begin{equation}
    A = \begin{mymatrix}{rrr}
      2  & 0 & 3 \\
      2  & 1 & 2 \\
      -6 & 0 & -7 \\
    \end{mymatrix}.
  \end{equation}
\end{example}

\begin{solution}
  The characteristic polynomial is
  \begin{eqnarray*}
    \det(A-\lambda I) \\
    &=&
        \begin{absmatrix}{ccc}
          2-\lambda & 0 & 3 \\
          2  & 1-\lambda & 2 \\
          -6 & 0 & -7-\lambda \\
        \end{absmatrix} \\
    &=&
        (2-\lambda)\begin{absmatrix}{ccc}
          1-\lambda & 2 \\
          0 & -7-\lambda \\
        \end{absmatrix}
    + 3 \begin{absmatrix}{ccc}
          2  & 1-\lambda \\
          -6 & 0 \\
        \end{absmatrix} \\
    &=& (2-\lambda)(1-\lambda)(-7-\lambda) + 18(1-\lambda) \\
    &=& -\lambda^3 -4\lambda^2 +\lambda + 4.
  \end{eqnarray*}
\end{solution}

It is time to summarize the method for finding the eigenvalues and
eigenvectors of a matrix.

\begin{procedure}{Finding eigenvalues and eigenvectors}{find-eigenvalues-vectors}
  \index{eigenvalue!calculating}%
  \index{eigenvector!calculating}%
  \index{matrix!eigenvalue!calculating}%
  \index{matrix!eigenvector!calculating}%
  Let $A$ be an $n\times n$-matrix. To find the eigenvalues and
  eigenvectors of $A$:
  \begin{enumerate}
  \item Calculate the characteristic polynomial $\det(A-\lambda I)$.
  \item The eigenvalues are the roots of the characteristic polynomial.
  \item For each eigenvalue $\lambda$, find a basis for the
    eigenvectors by solving the homogeneous system
    \begin{equation*}
      (A-\lambda I)\vect{v} = \vect{0}.
    \end{equation*}
  \end{enumerate}
  To double-check your work, make sure that $A\vect{v}=\lambda\vect{v}$
  for each eigenvalue $\lambda$ and associated eigenvector $\vect{v}$.
\end{procedure}

\begin{example}{Finding eigenvalues and eigenvectors}{finding-eigenvalues-eigenvectors}
  Find the eigenvalues and eigenvectors of the matrix
  \begin{equation}
    A = \begin{mymatrix}{rrr}
      2  & 0 & 3 \\
      2  & 1 & 2 \\
      -6 & 0 & -7 \\
    \end{mymatrix}.
  \end{equation}
\end{example}

\begin{solution}
  We already found the characteristic polynomial in
  Example~\ref{exa:characteristic-polynomial}. It is
  \begin{equation*}
    p(\lambda) = \det(A-\lambda I) = -\lambda^3 -4\lambda^2 +\lambda + 4.
  \end{equation*}
  Finding the roots of a cubic polynomial can be a bit tricky, but
  with some trial and error, we can find that $\lambda=1$ is a
  root. We can therefore factor out $(\lambda-1)$:
  \begin{equation*}
    p(\lambda) = (\lambda-1)(-\lambda^2-5\lambda-4).
  \end{equation*}
  Then we can use the quadratic formula to find the remaining two
  roots:
  \begin{equation*}
    \lambda = \frac{5\pm\sqrt{25-16}}{-2},
  \end{equation*}
  which yields the two roots $\lambda=-4$ and $\lambda=-1$. Therefore,
  we have
  \begin{equation*}
    p(\lambda) = -(\lambda-1)(\lambda+4)(\lambda+1),
  \end{equation*}
  and the eigenvalues of $A$ are $\lambda=1$, $\lambda=-1$, and
  $\lambda=-4$. We now find the eigenvectors for each eigenvalue.
  \begin{itemize}
  \item {\bf{\underline{For $\lambda=1$:}}} We must solve
    $(A-I)\vect{v}=\vect{0}$, i.e.,
    \begin{equation*}
      \begin{mymatrix}{rrr}
        1  & 0 & 3 \\
        2  & 0 & 2 \\
        -6 & 0 & -8 \\
      \end{mymatrix}\vect{v}=\vect{0}.
    \end{equation*}
    The basic solution is
    \begin{equation*}
      \vect{v}_1 = \begin{mymatrix}{r} 0 \\ 1 \\ 0 \end{mymatrix}.
    \end{equation*}
  \item {\bf{\underline{For $\lambda=-1$:}}} We must solve
    $(A+I)\vect{v}=\vect{0}$, i.e.,
    \begin{equation*}
      \begin{mymatrix}{rrr}
        3  & 0 & 3 \\
        2  & 2 & 2 \\
        -6 & 0 & -6 \\
      \end{mymatrix}\vect{v}=\vect{0}.
    \end{equation*}
    The basic solution is
    \begin{equation*}
      \vect{v}_2 = \begin{mymatrix}{r} 1 \\ 0 \\ -1 \end{mymatrix}.
    \end{equation*}
  \item {\bf{\underline{For $\lambda=-4$:}}} We must solve
    $(A+4I)\vect{v}=\vect{0}$, i.e.,
    \begin{equation*}
      \begin{mymatrix}{rrr}
        6  & 0 & 3 \\
        2  & 5 & 2 \\
        -6 & 0 & -3 \\
      \end{mymatrix}\vect{v}=\vect{0}.
    \end{equation*}
    The basic solution is
    \begin{equation*}
      \vect{v}_3 = \begin{mymatrix}{r} 5 \\ 2 \\ -10 \end{mymatrix}.
    \end{equation*}
  \end{itemize}
\end{solution}

\begin{example}{A zero eigenvalue}{zero-eigenvalue}
  Let
  \begin{equation*}
    A=\begin{mymatrix}{rrr}
      2 & 2 & -2 \\
      1 & 3 & -1 \\
      -1 & 1 & 1 \\
    \end{mymatrix}.
  \end{equation*}
  Find the eigenvalues and eigenvectors of $A$.
\end{example}

\begin{solution}
  To find the eigenvalues of $A$, we first compute the characteristic
  polynomial.
  \begin{equation*}
    \det(A-\lambda I) =
    \begin{absmatrix}{ccc}
      2-\lambda & 2 & -2 \\
      1 & 3-\lambda & -1 \\
      -1 & 1 & 1-\lambda \\
    \end{absmatrix}
    = -\lambda^{3}+6 \lambda^{2}-8\lambda.
  \end{equation*}
  You can verify that the roots of this polynomial are
  $\lambda_1 = 0$, $\lambda_2 = 2$, $\lambda_3 = 4$.  Notice that
  while eigenvectors can never equal $0$, it is possible to have an
  eigenvalue equal to $0$.  Now we will find the basic
  eigenvectors.
  \begin{itemize}
  \item {\bf{\underline{For $\lambda_1 =0$:}}} We must solve the
    equation $(A-0I)\vect{v} = \vect{0}$. This equation becomes
    $A\vect{v}=\vect{0}$. We write the augmented matrix for this
    system and reduce to echelon form:
    \begin{equation*}
      \begin{mymatrix}{rrr|r}
        2 & 2 & -2 & 0 \\
        1 & 3 & -1 & 0 \\
        -1 & 1 & 1 & 0 \\
      \end{mymatrix}
      \roweq\ldots\roweq
      \begin{mymatrix}{rrr|r}
        1 & 0 & -1 & 0 \\
        0 & 1 &  0 & 0 \\
        0 & 0 &  0 & 0 \\
      \end{mymatrix}.
    \end{equation*}
    The basic solution is
    \begin{equation*}
      \vect{v}_1
      =
      \begin{mymatrix}{r} 1 \\ 0 \\ 1 \end{mymatrix}.
    \end{equation*}
  \item {\bf{\underline{For $\lambda_2=2$:}}} We solve the
    equation $(A-2I)\vect{v} = \vect{0}$:
    \begin{equation*}
      \begin{mymatrix}{rrr|r}
        0  & 2 & -2 & 0 \\
        1  & 1 & -1 & 0 \\
        -1 & 1 & -1 & 0 \\
      \end{mymatrix}
      \roweq\ldots\roweq
      \begin{mymatrix}{rrr|r}
        1  & 0 &  0 & 0 \\
        0  & 1 & -1 & 0 \\
        0  & 0 &  0 & 0 \\
      \end{mymatrix}.
    \end{equation*}
    The basic solution is
    \begin{equation*}
      \vect{v}_2
      =
      \begin{mymatrix}{r} 0 \\ 1 \\ 1 \end{mymatrix}.
    \end{equation*}
  \item {\bf{\underline{For $\lambda_3=4$:}}} We solve the
    equation $(A-4I)\vect{v} = \vect{0}$:
    \begin{equation*}
      \begin{mymatrix}{rrr|r}
        -2 & 2  & -2 & 0 \\
        1  & -1 & -1 & 0 \\
        -1 &  1 & -3 & 0 \\
      \end{mymatrix}
      \roweq\ldots\roweq
      \begin{mymatrix}{rrr|r}
        1 & -1 & 0 & 0 \\
        0 &  0 & 1 & 0 \\
        0 &  0 & 0 & 0 \\
      \end{mymatrix}.
    \end{equation*}
    The basic solution is
    \begin{equation*}
      \vect{v}_3
      =
      \begin{mymatrix}{r} 1 \\ 1 \\ 0 \end{mymatrix}.
    \end{equation*}
  \end{itemize}
  Thus we have found the eigenvectors $\vect{v}_1$ for $\lambda_1$,
  $\vect{v}_2$ for $\lambda_2$, and $\vect{v}_3$ for $\lambda_3$.
  We can double-check our answers by checking the equation
  $A\vect{v}=\lambda\vect{v}$ in each case:
  \begin{eqnarray*}
    A\vect{v}_1
    &=&
    \begin{mymatrix}{rrr}
      2 & 2 & -2 \\
      1 & 3 & -1 \\
      -1 & 1 & 1 \\
    \end{mymatrix}
    \begin{mymatrix}{r} 1 \\ 0 \\ 1 \end{mymatrix}
    =
    \begin{mymatrix}{r} 0 \\ 0 \\ 0 \end{mymatrix}
    = 0\vect{v}_1,
    \\
    A\vect{v}_2
    &=&
    \begin{mymatrix}{rrr}
      2 & 2 & -2 \\
      1 & 3 & -1 \\
      -1 & 1 & 1 \\
    \end{mymatrix}
    \begin{mymatrix}{r} 0 \\ 1 \\ 1 \end{mymatrix}
    =
    \begin{mymatrix}{r} 0 \\ 2 \\ 2 \end{mymatrix}
    = 2\vect{v}_2,
    \\
    A\vect{v}_3
    &=&
    \begin{mymatrix}{rrr}
      2 & 2 & -2 \\
      1 & 3 & -1 \\
      -1 & 1 & 1 \\
    \end{mymatrix}
    \begin{mymatrix}{r} 1 \\ 1 \\ 0 \end{mymatrix}
    =
    \begin{mymatrix}{r} 4 \\ 4 \\ 0 \end{mymatrix}
    = 4\vect{v}_3.
  \end{eqnarray*}
  Therefore, our eigenvectors and eigenvalues are correct.
\end{solution}

