\section{Matrix multiplication}

Now that we have examined how to multiply a matrix by a vector, we
wish to consider the case where we multiply two matrices of more
general sizes, although these sizes still need to be appropriate as we
will see. For example, in Example \ref{exa:vector-mult-by-matrix}, we
multiplied a $3 \times 4$ matrix by a $4 \times 1$ vector.  We want to
investigate how to multiply other sizes of matrices.

We have not yet given any conditions on when matrix multiplication is
possible!  For matrices $A$ and $B$, in order to form the product
$AB$, the number of columns of $A$ must equal the number of rows of
$B$. Consider a product $AB$ where $A$ has size $m\times n$ and $B$
has size $n \times p$. Then, the product in terms of size of matrices
is given by
\begin{equation*}
  (m\times\overset{\text{these must match!}}{\widehat{n)\;(n}\times p})=m\times p
\end{equation*}

Note the two outside numbers give the size of the product. One of the
most important rules regarding matrix multiplication is the following.
If the two middle numbers don't match, you can't multiply the
matrices!

When the number of columns of $A$ equals the number of rows of $B$ the
two matrices are said to be
\textbf{conformable}\index{matrix!conformable} and the product
$AB$\index{matrix multiplication} is obtained as follows.

\begin{definition}{Multiplication of two matrices}{multiplication-of-two-matrices}
  Let $A$ be an $m\times n$ matrix and let $B$ be an $n\times p$
  matrix of the form
  \begin{equation*}
    B=\mat{B_{1} \cdots  B_{p}}
  \end{equation*}
  where $B_{1},...,B_{p}$ are the $n\times 1$ columns of $B$. Then the 
  $m\times p$ matrix $AB$ is defined as follows:
  \begin{equation*}
    AB = A \mat{B_{1} \cdots  B_{p}} =  \mat{(A B)_{1} \cdots  (AB)_{p}} 
  \end{equation*}
  where $(AB)_{k}$ is an $m\times 1$ matrix or column vector which
  gives the $k^{th}$ column of $AB$. 
\end{definition}

Consider the following example.

\begin{example}{Multiplying two matrices}{multiplication-of-two-matrices}
  Find $AB$ if possible.
  \begin{equation*}
    A = \begin{mymatrix}{rrr}
      1 & 2 & 1 \\
      0 & 2 & 1
    \end{mymatrix}, B = \begin{mymatrix}{rrr}
      1 & 2 & 0 \\
      0 & 3 & 1 \\
      -2 & 1 & 1
    \end{mymatrix}
  \end{equation*}
\end{example}

\begin{solution} The first thing you need to verify when calculating a
  product is whether the multiplication is possible.  The first matrix
  has size $2\times 3$ and the second matrix has size $3\times 3$. The
  inside numbers are equal, so $A$ and $B$ are conformable matrices.
  According to the above discussion $AB$ will be a $2\times 3$ matrix.
  Definition \ref{def:multiplication-of-two-matrices} gives us a way
  to calculate each column of $AB$, as follows.
  \begin{equation*}
    \mat{\overset{
        \text{First column}}{\overbrace{\begin{mymatrix}{rrr}
            1 & 2 & 1 \\
            0 & 2 & 1
          \end{mymatrix} \begin{mymatrix}{r}
            1 \\
            0 \\
            -2
          \end{mymatrix} }},\overset{\text{Second column}}{\overbrace{\begin{mymatrix}{rrr}
            1 & 2 & 1 \\
            0 & 2 & 1
          \end{mymatrix} \begin{mymatrix}{r}
            2 \\
            3 \\
            1
          \end{mymatrix} }},\overset{\text{Third column}}{\overbrace{\begin{mymatrix}{rrr}
            1 & 2 & 1 \\
            0 & 2 & 1
          \end{mymatrix} \begin{mymatrix}{r}
            0 \\
            1 \\
            1
          \end{mymatrix} }}}
  \end{equation*}
  You know how to multiply a matrix times a vector, using Definition
  \ref{def:multiplication-vector-matrix} for each of the three
  columns. Thus
  \begin{equation*}
    \begin{mymatrix}{rrr}
      1 & 2 & 1 \\
      0 & 2 & 1
    \end{mymatrix} \begin{mymatrix}{rrr}
      1 & 2 & 0 \\
      0 & 3 & 1 \\
      -2 & 1 & 1
    \end{mymatrix} =\allowbreak \begin{mymatrix}{rrr}
      -1 & 9 & 3 \\
      -2 & 7 & 3
    \end{mymatrix} 
  \end{equation*}
\end{solution}

Since vectors are simply $ n \times 1$ or $1 \times m$ matrices, we
can also multiply a vector by another vector.

\begin{example}{Vector times vector multiplication}{vector-multiplication}
  Multiply if possible $\begin{mymatrix}{r}
    1 \\
    2 \\
    1
  \end{mymatrix} \begin{mymatrix}{rrrr}
    1 & 2 & 1 & 0
  \end{mymatrix}$.
\end{example}

\begin{solution}
  In this case we are multiplying a matrix of size $3 \times 1$ by a
  matrix of size $1 \times 4$. The inside numbers match so the product
  is defined. Note that the product will be a matrix of size
  $3 \times 4$.  Using Definition
  \ref{def:multiplication-of-two-matrices}, we can compute this
  product as follows $\:$
  \begin{equation*}
    \begin{mymatrix}{r}
      1 \\
      2 \\
      1
    \end{mymatrix} \begin{mymatrix}{rrrr}
      1 & 2 & 1 & 0
    \end{mymatrix} = 
    \mat{\overset{
        \text{First column}}{\overbrace{\begin{mymatrix}{r}
            1 \\
            2 \\
            1
          \end{mymatrix} \begin{mymatrix}{r}
            1
          \end{mymatrix} }},\overset{\text{Second column}}{\overbrace{\begin{mymatrix}{r}
            1 \\
            2\\
            1
          \end{mymatrix} \begin{mymatrix}{r}
            2 
          \end{mymatrix} }},\overset{\text{Third column}}{\overbrace{\begin{mymatrix}{r}
            1 \\
            2 \\
            1
          \end{mymatrix} \begin{mymatrix}{r}
            1
          \end{mymatrix} }}, \overset {\text{Fourth column}}{\overbrace{\begin{mymatrix}{r}
            1\\
            2\\
            1
          \end{mymatrix} \begin{mymatrix}{r}
            0
          \end{mymatrix}}}
    }
  \end{equation*}

  You can use Definition \ref{def:multiplication-vector-matrix} to verify that this product is
  \begin{equation*}
    \begin{mymatrix}{cccc}
      1 & 2 & 1 & 0 \\
      2 & 4 & 2 & 0 \\
      1 & 2 & 1 & 0
    \end{mymatrix}
  \end{equation*}
\end{solution}

\begin{example}{A multiplication which is not defined}{undefined-matrix-multiplication}
  Find $BA$ if possible.
  \begin{equation*}
    B = \begin{mymatrix}{ccc}
      1 & 2 & 0 \\
      0 & 3 & 1 \\
      -2 & 1 & 1
    \end{mymatrix},  A = \begin{mymatrix}{ccc}
      1 & 2 & 1 \\
      0 & 2 & 1
    \end{mymatrix}
  \end{equation*}
\end{example}

\begin{solution}
  First check if it is possible. This product is of the form
  $\tup{3\times 3} \tup{2\times 3}$. The inside numbers do not match
  and so you can't do this multiplication.
\end{solution}

In this case, we say that the multiplication is not defined.  Notice
that these are the same matrices which we used in Example
\ref{exa:multiplication-of-two-matrices}.  In this example, we tried
to calculate $BA$ instead of $AB$. This demonstrates another property
of matrix multiplication. While the product $AB$ maybe be defined, we
cannot assume that the product $BA$ will be possible. Therefore, it is
important to always check that the product is defined before carrying
out any calculations.

Earlier, we defined the zero matrix $0$ to be the matrix (of
appropriate size) containing zeros in all entries.  Consider the
following example for multiplication by the zero matrix.

\begin{example}{Multiplication by the zero matrix}{mult-by-zero-matrix}
  Compute the product $A0$ for the matrix
  \begin{equation*}
    A=
    \begin{mymatrix}{rr}
      1 & 2 \\
      3 & 4
    \end{mymatrix}
  \end{equation*}
  and the $2 \times 2$ zero matrix given by
  \begin{equation*}
    0=
    \begin{mymatrix}{rr}
      0 & 0 \\
      0 & 0
    \end{mymatrix}
  \end{equation*}
\end{example}

\begin{solution} 
  In this product, we compute
  \begin{equation*}
    \begin{mymatrix}{rr}
      1 & 2 \\
      3 & 4
    \end{mymatrix}
    \begin{mymatrix}{rr}
      0 & 0 \\
      0 & 0
    \end{mymatrix}
    =
    \begin{mymatrix}{rr}
      0 & 0 \\
      0 & 0
    \end{mymatrix}
  \end{equation*}

  Hence, $A0=0$. 
\end{solution}

Notice that we could also multiply $A$ by the $2 \times 1 $ zero
vector given by $\begin{mymatrix}{r} 0 \\ 0 \end{mymatrix}$.  The
result would be the $2 \times 1$ zero vector.  Therefore, it is always
the case that $A0=0$, for an appropriately sized zero matrix or
vector.
