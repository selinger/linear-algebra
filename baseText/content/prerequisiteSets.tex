\section{Sets and set notation}

A set is a collection\index{set notation} of things called elements.
For example $\set{1,2,3,8} $
would be a set consisting of the elements 1,2,3, and 8. To indicate that $3$
is an element of $\set{1,2,3,8}$, it is customary to write $3\in
\set{1,2,3,8}$. We can also indicate when an element is not in a set, 
by writing $9\notin \set{1,2,3,8} $ which says that $9$ is
not an element of $\set{1,2,3,8}$. Sometimes a rule specifies a
set. For example you could specify a set as all integers larger than $2$.
This would be written as $S=\set{x\in \Z:x>2}$. This
notation says: $S$ is the set of all integers, $x$, such that $x>2$.

Suppose $A$ and $B$ are sets with the property that every element of $A$ is an
element of $B$. Then we say that $A$ is a subset of $B$. For example, $\set{
1,2,3,8} $ is a subset of $\set{1,2,3,4,5,8}$. In symbols, we write
$\set{1,2,3,8} \subseteq \set{1,2,3,4,5,8}$. It is
sometimes said that ``$A$ is contained in $B$" or even ``$B$ contains $A$".
The same statement about the two sets may also be written as $\set{
1,2,3,4,5,8} \supseteq \set{1,2,3,8}$.

We can also talk about the {\em union\em}\index{union} of two sets, which we write as $A \cup B$. This is the set consisting of everything which is an
element of at least one of the sets, $A$ or $B$. As an example of the union
of two sets, consider $\set{1,2,3,8} \cup \set{3,4,7,8} =\set{
1,2,3,4,7,8}$. This set is made up of the numbers which are in at least
one of the two sets.\index{$\cup$}

In general
\begin{equation*}
A\cup B = \set{x:x\in A
\text{ or }x\in B} 
\end{equation*}
Notice that an element which is in {\em both\em} $A$ and $B$ is also in the
union, as well as elements which are in only one of $A$ or $B$. 

Another important set is the intersection\index{intersection}\index{$\cap$} of two sets $A$ and $B$, written $A \cap B$. This set consists of everything which is in
{\em both\em} of the sets. Thus $\set{1,2,3,8} \cap \set{3,4,7,8}
=\set{3,8} $ because $3$ and $8$ are those elements the two sets
have in common. In general,
\begin{equation*}
A\cap B =  \set{x:x\in A\text{ and }x\in B} 
\end{equation*}

If $A$ and $B$ are two sets, $A\setminus B$\index{$\setminus$} denotes the set of things which
are in $A$ but not in $B$. Thus
\begin{equation*}
A\setminus B =  \set{x\in A:x\notin B} 
\end{equation*}
For example, if $A = \set{1,2,3,8 }$ and $B = \set{3,4,7,8 }$, then $A \setminus B = \set{1,2,3,8} \setminus 
\set{3,4,7,8 } =\set{1,2 }$.

A special set which is very important in mathematics is the empty set\index{empty set} denoted by $\emptyset$. The empty set, $\emptyset$, is
defined as the set which has no elements in it. It follows that the empty set is a subset of every set. 
This is true because if it were not so, there would have to exist a set $A$, such that $\emptyset $
has something in it which is not in $A$. However, $\emptyset $ has nothing
in it and so it must be that $\emptyset \subseteq A$.

We can also use brackets to denote sets which are intervals of numbers. Let $a$ and $b$ be real numbers. Then

\begin{itemize}
\item $[a,b] = \set{x \in \R \mid a\leq x\leq b}$.
\item $[a,b) = \set{x \in \R \mid a\leq x<b}$.
\item $(a,b] = \set{x \in \R \mid a<x\leq b}$.
\item $(a,b) = \set{x \in \R \mid a<x<b}$.
\item $[a,\infty) = \set{x \in \R \mid a\leq x}$.
\item $(-\infty,a] = \set{x \in \R \mid x\leq a}$.
\item $(a,\infty) = \set{x \in \R \mid a<x}$.
\item $(-\infty,a) = \set{x \in \R \mid x<a}$.
\end{itemize}

These sorts of sets of real
numbers are called intervals. The two points $a$ and $b$ are called
endpoints, or bounds, of the interval. In particular, $a$ is the {\em lower bound \em}  while $b$ is the {\em upper bound \em} of the above
intervals, where applicable.\index{intervals!notation} Other intervals such as $\tup{-\infty ,b} $
are defined by analogy to what was just explained.
 In general, the curved
parenthesis, $($, indicates the end point is not included in the interval, while
the square parenthesis, $[$, indicates this end point is included. The reason that
there will always be a curved parenthesis next to $\infty $ or $-\infty $ is
that these are not real numbers and cannot be included in the interval in the way a real number can. 

To illustrate the use of this notation relative to intervals consider three
examples of inequalities. Their solutions will be written in the interval notation
just described.

\begin{example}{Solving an inequality}{solving-inequality1}
Solve the inequality $2x+4\leq x-8$.
\end{example}

\begin{solution}
We need to find $x$ such that $2x+4\leq x-8$. Solving for $x$, we see that 
$x\leq -12$ is the answer. This is written in terms of an interval as $(-\infty ,-12]$.
\end{solution}

Consider the following example.

\begin{example}{Solving an inequality}{solving-inequality2}
Solve the inequality $\tup{x+1} \tup{2x-3} \geq0$.
\end{example}

\begin{solution}
We need to find $x$ such that $\tup{x+1} \tup{2x-3} \geq0$. 
The solution is given by  $x\leq -1$ or $x\geq \frac{3}{2}$. Therefore, 
$x$ which fit into either of these intervals gives a solution. In terms of set notation this is denoted by $(-\infty ,-1]\cup
[ \vspace{0.05in}\frac{3}{2},\infty )$.
\end{solution}

Consider one last example.

\begin{example}{Solving an inequality}{solving-inequality3}
Solve the inequality $x \tup{x+2} \geq-4$.
\end{example}

\begin{solution}
This inequality is true for any value of $x$ where $x$ is a real number. We can write the solution as $\R$ or $\tup{
-\infty ,\infty }$.
\end{solution}

In the next section, we examine another important mathematical concept.
